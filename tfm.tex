\documentclass[b5paper]{book}
\usepackage{hyperref}
\usepackage{makeidx}
\usepackage{amssymb}
\usepackage{color}
\usepackage{alltt}
\usepackage{graphicx}
\usepackage{layout}
\def\union{\cup}
\def\intersect{\cap}
\def\getsrandom{\stackrel{\rm R}{\gets}}
\def\cross{\times}
\def\cat{\hspace{0.5em} \| \hspace{0.5em}}
\def\catn{$\|$}
\def\divides{\hspace{0.3em} | \hspace{0.3em}}
\def\nequiv{\not\equiv}
\def\approx{\raisebox{0.2ex}{\mbox{\small $\sim$}}}
\def\lcm{{\rm lcm}}
\def\gcd{{\rm gcd}}
\def\log{{\rm log}}
\def\ord{{\rm ord}}
\def\abs{{\mathit abs}}
\def\rep{{\mathit rep}}
\def\mod{{\mathit\ mod\ }}
\renewcommand{\pmod}[1]{\ ({\rm mod\ }{#1})}
\newcommand{\floor}[1]{\left\lfloor{#1}\right\rfloor}
\newcommand{\ceil}[1]{\left\lceil{#1}\right\rceil}
\def\Or{{\rm\ or\ }}
\def\And{{\rm\ and\ }}
\def\iff{\hspace{1em}\Longleftrightarrow\hspace{1em}}
\def\implies{\Rightarrow}
\def\undefined{{\rm ``undefined"}}
\def\Proof{\vspace{1ex}\noindent {\bf Proof:}\hspace{1em}}
\let\oldphi\phi
\def\phi{\varphi}
\def\Pr{{\rm Pr}}
\newcommand{\str}[1]{{\mathbf{#1}}}
\def\F{{\mathbb F}}
\def\N{{\mathbb N}}
\def\Z{{\mathbb Z}}
\def\R{{\mathbb R}}
\def\C{{\mathbb C}}
\def\Q{{\mathbb Q}}
\definecolor{DGray}{gray}{0.5}
\newcommand{\emailaddr}[1]{\mbox{$<${#1}$>$}}
\def\twiddle{\raisebox{0.3ex}{\mbox{\tiny $\sim$}}}
\def\gap{\vspace{0.5ex}}
\makeindex
\begin{document}
\frontmatter
\pagestyle{empty}
\title{TomsFastMath User Manual \\ v0.13.1}
\author{Tom St Denis \\ tomstdenis@gmail.com}
\maketitle
This text and library are all hereby placed in the public domain.  This book has been formatted for B5 
[176x250] paper using the \LaTeX{} {\em book} macro package.

\vspace{13cm}

\begin{flushleft}This project was sponsored in part by  

Secure Science Corporation \url{http://www.securescience.net}.
\end{flushleft}

\tableofcontents
\listoffigures
\mainmatter
\pagestyle{headings}
\chapter{Introduction}
\section{What is TomsFastMath?}

TomsFastMath is meant to be a very fast yet still fairly portable and easy to port large
integer arithmetic library written in ISO C.  The goal specifically is to be able to perform
very fast modular exponentiations and other related functions required for ECC, DH and RSA
cryptosystems.

Most of the library is pure ISO C portable source code while a small portion (three files) contain
a mixture of ISO C and assembler inline fragments.  Compared to LibTomMath this new library is 
meant to be much faster while sacrificing flexibiltiy.  This is accomplished through several means.

\begin{enumerate}
   \item The new code is slightly messier and contains asm blocks.
   \item This uses fixed not multiple precision integers.  
   \item It is designed only for fast modular exponentiations [e.g. less flexibility].
\end{enumerate}

To mitigate some of the problems that arise from using assembler it has been carefully and 
appropriately used where it would make the most gain in performance.  Also we use macro's
for assembler code which allows new ports to be inserted easily.

The new code uses fixed precision arithmetic which means at compile time you choose a maximum 
precision and all numbers are limited to that.  This has the benefit of not requiring any
memory heap operations (which are slow) in any of the functions.  It has the downside that 
integers that are too large are truncated.

The goal of this library is to be able to perform modular exponentiations (with an odd modulus) very
fast.  This is what takes the most time in systems such as RSA and DH.  This also requires
fast multiplication and squaring and has the side effect of speeding up ECC operations as well.

\section{License}
TomsFastMath is public domain.

\section{Building}
To build the library simply type ``make''.  Or to install in typical *unix like directories use
``make install''.  Similarly a shared library can be built with ``make -f makefile.shared install''.

You can build the test program with ``make test''.  To perform simple static testing (useful to 
test out new assembly ports) use the stest program.  Type ``make stest'' and run it on your 
target.  The program will perform three multiplications, squarings and montgomery reductions.  
Likely if your assembly code is invalid this code will exhibit the bug.

\subsection{Intel CC}
In theory you should be able to build the library with

\begin{verbatim}
CFLAGS="-O3 -ip" CC=icc make IGNORE_SPEED=1
\end{verbatim}

However, Intels inline assembler is way less advanced than GCCs.  As a result it doesn't compile.  
Fortunately it doesn't really matter.

\subsection{MSVC}
The library doesn't build with MSVC.  Imagine that.

\subsection{Build Limitations}
TomsFastMath has the following build requirements which are non--portable but under most 
circumstances not problematic.

\begin{enumerate}
\item ``CHAR\_BIT'' must be eight.  
\item The ``fp\_digit'' type must be a multiple of eight bits long.
\item The ``fp\_word'' must be at least twice the length of fp\_digit.
\end{enumerate}

\subsection{Optimization Configuration}
By default TFM is configured for 32--bit digits using ISO C source code.  This mode while portable
is not very efficient.  While building the library (from scratch) you can define one of 
several ``CFLAGS'' defines.

For example, to build with with SSE2 optimizations type 

\begin{verbatim}
CFLAGS=-DTFM_SSE2 make clean libtfm.a
\end{verbatim}

\subsubsection{x86--32}  The ``x86--32'' mode is defined by ``TFM\_X86'' and covers all
i386 and beyond processors.  It requires GCC to build and only works with 32--bit digits.  In this 
mode fp\_digit is 32--bits and fp\_word is 64--bits.  This mode will be autodetected when building 
with GCC to an  ``i386'' target.  You can override this behaviour by defining TFM\_NO\_ASM or 
another optimization mode (such as SSE2).

\subsubsection{SSE2} The ``SSE2'' mode is defined by ``TFM\_SSE2'' and requires a Pentium 4, Pentium
M or Athlon64 processor.  It requires GCC to build.  Note that you shouldn't define both
TFM\_X86 and TFM\_SSE2 at the same time.   This mode only works with 32--bit digits.  In this 
mode fp\_digit is 32--bits and fp\_word is 64--bits.  While this mode will work on the AMD Athlon64 
series of processors it is less efficient than the native ``x86--64'' mode and not recommended.

There is an additional ``TFM\_PRESCOTT'' flag that you can define for P4 Prescott processors.  This causes
the mul/sqr functions to use x86\_32 and the montgomery reduction to use SSE2 which is (so far) the fastest
combination.  If you are using an older (e.g. Northwood) generation P4 don't define this.

\subsubsection{x86--64}  The ``x86--64'' mode is defined by ``TFM\_X86\_64'' and requires a 
``x86--64'' capable processor (Athlon64 and future Pentium processors).  It requires GCC to
build and only works with 64--bit digits.  Note that by enabling this mode it will automatically
enable 64--bit digits.  In this mode fp\_digit is 64--bits and fp\_word is 128--bits.  This mode will
be autodetected when building with GCC to an ``x86--64'' target.  You can override this behaviour by defining
TFM\_NO\_ASM.

\subsubsection{ARM}  The ``ARM'' mode is defined by ``TFM\_ARM'' and requires a ARMv4 with the M instructions (enhanced 
multipliers) or higher processor.  It requires GCC and works with 32--bit digits.  In this mode fp\_digit is 32--bits and 
fp\_word is 64--bits.

\subsubsection{PPC32} The ``PPC32'' mode is defined by ``TFM\_PPC32'' and requires a standard PPC processor.  It doesn't 
use altivec or other extensions so it should work on all compliant implementations of PPC.  It requires GCC and works
with 32--bit digits.  In this mode fp\_digit is 32--bits and fp\_word is 64--bits.

\subsubsection{PPC64} The ``PPC64'' mode is defined by ``TFM\_PPC64'' and requires a 64--bit PPC processor.  

\subsubsection{AVR32} The ``AVR32'' mode is defined by ``TFM\_AVR32'' and requires an Atmel AVR32 processor.  

\subsubsection{Future Releases}  Future releases will support additional platform optimizations.
Developers of MIPS and SPARC platforms are encouraged to submit GCC asm inline patches 
(see chapter \ref{chap:asmops} for more information).

\begin{figure}[here]
\begin{small}
\begin{center}
\begin{tabular}{|l|l|}
\hline \textbf{Processor} & \textbf{Recommended Mode} \\
\hline All 32--bit x86 platforms  & TFM\_X86 \\
\hline Pentium 4                  & TFM\_SSE2 \\
\hline Pentium 4 Prescott         & TFM\_SSE2 + TFM\_PRESCOTT \\
\hline Athlon64                   & TFM\_X86\_64 \\
\hline ARMv4 or higher with M     & TFM\_ARM \\
\hline G3/G4 (32-bit PPC)         & TFM\_PPC32 \\
\hline G5 (64-bit PPC)            & TFM\_PPC64 \\
\hline Atmel AVR32                & TFM\_AVR32 \\
\hline &\\
\hline x86--32 or x86--64 (with GCC) & Leave blank and let autodetect work \\
\hline
\end{tabular}
\caption{Recommended Build Modes}
\end{center}
\end{small}
\end{figure}

\subsection{Build Configurations}
TomsFastMath is configurable in terms of which unrolled code (if any) is included.  By default, the majority of the code is included which
results in large binaries.  The first flag to try out is TFM\_ALREADY\_SET which tells TFM to turn off \textbf{all} unrolled code.  This will
result in a smaller library but also a much slower library.

From this clean state, you can start enabling unrolled code for given cryptographic tasks at hand.  A series of TFM\_MULXYZ and TFM\_SQRXYZ macros
exist to enable specific unrolled code.  For instance, TFM\_MUL32 will enable a 32 digit unrolled multiplier.  For a complete list see the tfm.h header
file.  Keep in mind this is for digits not bits.  For example, you should enable TFM\_MUL16 if you are doing 1024-bit exptmods on a 64--bit platform, enable
TFM\_MUL32 on 32--bit platforms.

To help developers use ECC there are a set of defines for the five NIST curve sizes.  They are named TFM\_ECCXYZ where XYZ is one of 192, 224, 256, 384, or 521.  These
enable the multipliers and squaring code for a given curve, autodetecting 64--bit platforms as well.  

\subsection{Precision Configuration}
The precision of all integers in this library are fixed to a limited precision.  Essentially
the rule of setting the precision is if you plan on doing modular exponentiation with $k$--bit
numbers than the precision must be fixed to $2k$--bits plus four digits.  

This is changed by altering the value of ``FP\_MAX\_SIZE'' in tfm.h to your desired size.  By default, 
the library is configured to handle upto 2048--bit inputs to the modular exponentiator.  

\chapter{Getting Started}
\section{Data Types}
TomsFastMath is a large fixed precision integer library.  It provides the functionality to 
manipulate large signed integers through a relatively trivial api and a single data type.

The ``fp\_int'' or fixed precision integer is the data type that the functions operate with.  

\begin{verbatim}
typedef struct {
    fp_digit dp[FP_SIZE];
    int      used, 
             sign;
} fp_int;
\end{verbatim}

The \textbf{dp} member is the array of digits that forms the number.  It must always be zero 
padded.  The \textbf{used} member is the count of digits used in the array.  Although the 
precision is fixed the algorithms are still tuned to not process the entire array if it 
does not have to.  The \textbf{sign} indicates the sign of the integer.  It is \textbf{FP\_ZPOS} (0)
if the integer is zero or positive and \textbf{FP\_NEG} (1) otherwise.

\section{Initialization}
\subsection{Simple Initialization}
To initialize an integer to the default state of zero use the fp\_init() function.

\index{fp\_init}
\begin{verbatim}
void fp_init(fp_int *a);
\end{verbatim}

This will initialize the fp\_int $a$ to zero.  Note that the function fp\_zero() is an alias
for fp\_init().

\subsection{Initialize Small Constants}
To initialize an integer with a small single digit value use the fp\_set() function.

\index{fp\_set}
\begin{verbatim}
void fp_set(fp_int *a, fp_digit b);
\end{verbatim}

This will initialize $a$ and set it equal to the digit $b$.  

\subsection{Initialize Copy}
To initialize an integer with a copy of another integer use the fp\_init\_copy() function.

\index{fp\_init\_copy}
\begin{verbatim}
void fp_init_copy(fp_int *a, fp_int *b)
\end{verbatim}

This will initialize $a$ as a copy of $b$.  Note that for compatibility with LibTomMath the function
fp\_copy() is also provided.

\subsection{Initialization with a random value}
To initialize an integer with a random value of a specific length use the fp\_rand() function.

\index{fp\_rand}
\begin{verbatim}
void fp_rand(fp_int *a, int digits)
\end{verbatim}

This will initialize $a$ with $digits$ random digits.

\index{FP\_GEN\_RANDOM} \index{FP\_GEN\_RANDOM\_MAX}
The source of the random data is \textbf{arc4random()} on *BSD systems that provide this function
and the standard C function \textbf{rand()} on all other systems. It can be configured at compile time
by pre-defining \textbf{FP\_GEN\_RANDOM} and \textbf{FP\_GEN\_RANDOM\_MAX}.

\chapter{Arithmetic Operations}
\section{Odds and Evens}
To quickly and easily tell if an integer is zero, odd or even use the following functions.

\index{fp\_iszero} \index{fp\_iseven} \index{fp\_isodd}
\begin{verbatim}
int fp_iszero(fp_int *a);
int fp_iseven(fp_int *a);
int fp_isodd(fp_int *a);
\end{verbatim}

These will return \textbf{FP\_YES} if the answer to their respective questions is yes.  Otherwise they
return \textbf{FP\_NO}.  Note that these are implemented as macros and as such you should avoid using 
++ or --~-- operators on the input operand.

\section{Sign Manipulation}
To negate or compute the absolute of an integer use the following functions.

\index{fp\_neg} \index{fp\_abs}
\begin{verbatim}
void fp_neg(fp_int *a, fp_int *b);
void fp_abs(fp_int *a, fp_int *b);
\end{verbatim}
This will compute the negation (or absolute) of $a$ and store the result in $b$.  Note that these 
are implemented as macros and as such you should avoid using ++ or --~-- operators on the input 
operand.

\section{Comparisons}
To perform signed or unsigned comparisons use following functions.

\index{fp\_cmp} \index{fp\_cmp\_mag}
\begin{verbatim}
int fp_cmp(fp_int *a, fp_int *b);
int fp_cmp_mag(fp_int *a, fp_int *b);
\end{verbatim}
These will compare $a$ to $b$.  They will return \textbf{FP\_GT} if $a$ is larger than $b$, 
\textbf{FP\_EQ} if they are equal and \textbf{FP\_LT} if $a$ is less than $b$.

The function fp\_cmp performs signed comparisons while the other performs unsigned comparisons.

\section{Shifting}
To shift the digits of an fp\_int left or right use the following functions.

\index{fp\_lshd} \index{fp\_rshd}
\begin{verbatim}
void fp_lshd(fp_int *a, int x);
void fp_rshd(fp_int *a, int x);
\end{verbatim}

These will shift the digits of $a$ left (or right respectively) $x$ digits.  

To shift individual bits of an fp\_int use the following functions.

\index{fp\_div\_2d} \index{fp\_mod\_2d} \index{fp\_mul\_2d} \index{fp\_div\_2} \index{fp\_mul\_2}
\begin{verbatim}
void fp_div_2d(fp_int *a, int b, fp_int *c, fp_int *d);
void fp_mod_2d(fp_int *a, int b, fp_int *c);
void fp_mul_2d(fp_int *a, int b, fp_int *c);
void fp_mul_2(fp_int *a, fp_int *c);
void fp_div_2(fp_int *a, fp_int *c);
void fp_2expt(fp_int *a, int b);
\end{verbatim}
fp\_div\_2d() will divide $a$ by $2^b$ and store the quotient in $c$ and remainder in $d$.  Either of 
$c$ or $d$ can be \textbf{NULL} if their value is not required.  fp\_mod\_2d() is a shortcut to 
compute the remainder directly.  fp\_mul\_2d() will multiply $a$ by $2^b$ and store the result in $c$.  

The fp\_mul\_2() and fp\_div\_2() functions are optimized multiplication and divisions by two.  The 
function fp\_2expt() will compute $a = 2^b$ quickly.

To quickly count the number of least significant bits that are zero use the following function.

\index{fp\_cnt\_lsb}
\begin{verbatim}
int fp_cnt_lsb(fp_int *a);
\end{verbatim}
This will return the number of adjacent least significant bits that are zero.  This is equivalent 
to the number of times two evenly divides $a$.

\section{Basic Algebra}

The following functions round out the basic algebraic functionality of the library.

\index{fp\_add} \index{fp\_sub} \index{fp\_mul} \index{fp\_sqr} \index{fp\_div} \index{fp\_mod}
\begin{verbatim}
void fp_add(fp_int *a, fp_int *b, fp_int *c);
void fp_sub(fp_int *a, fp_int *b, fp_int *c);
void fp_mul(fp_int *a, fp_int *b, fp_int *c);
void fp_sqr(fp_int *a, fp_int *b);
int fp_div(fp_int *a, fp_int *b, fp_int *c, fp_int *d);
int fp_mod(fp_int *a, fp_int *b, fp_int *c);
\end{verbatim}

The functions fp\_add(), fp\_sub() and fp\_mul() perform their respective operations on $a$ and
$b$ and store the result in $c$.  The function fp\_sqr() computes $b = a^2$ and is faster than
using fp\_mul() to perform the same operation.

The function fp\_div() divides $a$ by $b$ and stores the quotient in $c$ and remainder in $d$.  Either 
of $c$ and $d$ can be \textbf{NULL} if the result is not required.  The function fp\_mod() is a simple 
shortcut to find the remainder.

\section{Modular Exponentiation}
To compute a modular exponentiation use the following function.

\index{fp\_exptmod}
\begin{verbatim}
int fp_exptmod(fp_int *a, fp_int *b, fp_int *c, fp_int *d);
\end{verbatim}
This computes $d \equiv a^b \mbox{ (mod }c\mbox{)}$ for any odd $c$ and $b$.  $b$ may be negative so long as 
$a^{-1} \mbox{ (mod }c\mbox{)}$ exists.  The initial value of $a$ may be larger than $c$.  The size of $c$ must be 
half of the maximum precision used during the build of the library.  For example, by default $c$ must be less 
than $2^{2048}$.  

\section{Number Theoretic}

To perform modular inverses, greatest common divisor or least common multiples use the following
functions.

\index{fp\_invmod} \index{fp\_gcd} \index{fp\_lcm}
\begin{verbatim}
int fp_invmod(fp_int *a, fp_int *b, fp_int *c);
void fp_gcd(fp_int *a, fp_int *b, fp_int *c);
void fp_lcm(fp_int *a, fp_int *b, fp_int *c);
\end{verbatim}

The fp\_invmod() function will find the modular inverse of $a$ modulo an odd modulus $b$ and store
it in $c$ (provided it exists).  The function fp\_gcd() will compute the greatest common
divisor of $a$ and $b$ and store it in $c$.  Similarly the fp\_lcm() function will compute
the least common multiple of $a$ and $b$ and store it in $c$.

\section{Prime Numbers}
To quickly test a number for primality call this function.

\index{fp\_isprime}
\index{fp\_isprime\_ex}
\begin{verbatim}
int fp_isprime_ex(fp_int *a, int t);
\end{verbatim}
This will return \textbf{FP\_YES} if $a$ is probably prime.  It uses 256 trial divisions and
$t$ rounds of Rabin-Miller testing.  Note that this routine performs modular exponentiations
which means that $a$ must be in a valid range of precision.
The valid range of $t$ is $1 \ldots 256$.

For backwards compatibility reasons the API function fp\_isprime(a) is still available
and simply calls fp\_isprime\_ex(a, 8).

\chapter{Porting TomsFastMath}
\label{chap:asmops}
\section{Getting Started}
Porting TomsFastMath to a given processor target is usually a simple procedure.  For the most part 
assembly is used to get around the lack of a ``add with carry'' operation in the C language.  To
make matters simpler the use of assembler is through macro blocks.

Each ``port'' is defined by a block of code that re-defines the portable ISO C macros with assembler
inline blocks.  To add a new port you must designate a TFM\_XXX define that will enable your 
port when built.

\section{Multiply with Comba}
The file ``fp\_mul\_comba.c'' is responsible for providing the fast multiplication within the 
library.  This comba multiplication is fairly simple.  It uses a sliding three digit carry 
system with the variables $c0$, $c1$, $c2$.  For every digit of output $c0$ is the what will
be that digit, $c1$ will carry into the next digit and $c2$ will be the ``c1'' carry for
the next digit.  For every ``next'' digit effectively $c0$ is stored as output, $c1$ moves into
$c0$, $c2$ into $c1$ and zero into $c2$.

The following macros define the assmebler interface to the code.

\begin{verbatim}
#define COMBA_START 
\end{verbatim}

This is issued at the beginning of the multiplication function.  This is in place to allow you to
initialize any registers or machine words required.  You can leave it blank if you do not need 
it.

\begin{verbatim}
#define COMBA_CLEAR \
   c0 = c1 = c2 = 0;
\end{verbatim}

This clears the three comba carries.  If you are going to place carries in registers then 
zero the appropriate registers.  Note that the functions do not use $c0$, $c1$ or $c2$ directly
so you are free to ignore these varibles and use registers directly.

\begin{verbatim}
#define COMBA_FORWARD \
   c0 = c1; c1 = c2; c2 = 0;
\end{verbatim}

This propagates the carries after a digit has been produced.  

\begin{verbatim}
#define COMBA_STORE(x) \
   x = c0;
\end{verbatim}

This stores the $c0$ digit in the memory location specified by $x$.  Note that if you manually 
aliased $c0$ with a register than just store that register in $x$.  

\begin{verbatim}
#define COMBA_STORE2(x) \
   x = c1;
\end{verbatim}

This stores the $c1$ digit in the memory location specified by $x$.  Note that if you manually 
aliased $c1$ with a register than just store that register in $x$.  

\begin{verbatim}
#define COMBA_FINI
\end{verbatim}

If at the end of the function you need to perform some action fill this macro in. 

\begin{verbatim}
#define MULADD(i, j)                                          \
   t  = ((fp_word)i) * ((fp_word)j);                          \
   c0 = (c0 + t);              if (c0 < ((fp_digit)t))  ++c1; \
   c1 = (c1 + (t>>DIGIT_BIT)); if (c1 < (t>>DIGIT_BIT)) ++c2;
\end{verbatim}

This macro performs the ``multiply and add'' step that is central to the comba
multiplier.  It multiplies the fp\_digits $i$ and $j$ to produce a fp\_word result.  Effectively
the double--digit value is added to the three-digit carry formed by $c0$, $c1$, $c2$ where $c0$
is the least significant digit.

\section{Squaring with Comba}
Squaring is similar to multiplication except that it uses a special ``multiply and add twice'' macro
that replaces multiplications that are not required.

\begin{verbatim}
#define COMBA_START
\end{verbatim}

This allows for any initialization code you might have.  

\begin{verbatim}
#define CLEAR_CARRY \
   c0 = c1 = c2 = 0;
\end{verbatim}

This will clear the carries.  Like multiplication you can safely alias the three carry variables
to registers if you can/want to.

\begin{verbatim}
#define COMBA_STORE(x) \
   x = c0;
\end{verbatim}

Store the $c0$ carry to a given memory location.

\begin{verbatim}
#define COMBA_STORE2(x) \
   x = c1;
\end{verbatim}

Store the $c1$ carry to a given memory location.

\begin{verbatim}
#define CARRY_FORWARD \
   c0 = c1; c1 = c2; c2 = 0;
\end{verbatim}

Forward propagate all three carry variables.

\begin{verbatim}
#define COMBA_FINI
\end{verbatim}

If you need to clean up at the end of the function.

\begin{verbatim}
/* multiplies point i and j, updates carry "c1" and digit c2 */
#define SQRADD(i, j)                       \
   t  = ((fp_word)i) * ((fp_word)j);       \
   c0 = (c0 + t);              if (c0 < ((fp_digit)t))  ++c1; \
   c1 = (c1 + (t>>DIGIT_BIT)); if (c1 < (t>>DIGIT_BIT)) ++c2; 
\end{verbatim}

This is essentially the MULADD macro from the multiplication code.

\begin{verbatim}
/* for squaring some of the terms are doubled... */
#define SQRADD2(i, j)                       \
   t  = ((fp_word)i) * ((fp_word)j);       \
   c0 = (c0 + t);              if (c0 < ((fp_digit)t))  ++c1; \
   c1 = (c1 + (t>>DIGIT_BIT)); if (c1 < (t>>DIGIT_BIT)) ++c2; \
   c0 = (c0 + t);              if (c0 < ((fp_digit)t))  ++c1; \
   c1 = (c1 + (t>>DIGIT_BIT)); if (c1 < (t>>DIGIT_BIT)) ++c2; 
\end{verbatim}

This is like SQRADD except it adds the produce twice.  It's similar to 
computing SQRADD(i, j*2).

To further make things interesting the squaring code also has ``doubles'' (see my LTM book chapter five...) which are
handled with these macros.

\begin{verbatim}
#define SQRADDSC(i, j)                                                         \
   do { fp_word t;                                                             \
      t =  ((fp_word)i) * ((fp_word)j);                                        \
      sc0 = (fp_digit)t; sc1 = (t >> DIGIT_BIT); sc2 = 0;                      \
   } while (0);
\end{verbatim}
This computes a product and stores it in the ``secondary'' carry registers $\left < sc0, sc1, sc2 \right >$.

\begin{verbatim}
#define SQRADDAC(i, j)                                                         \
   do { fp_word t;                                                             \
   t = sc0 + ((fp_word)i) * ((fp_word)j);  sc0 = t;                            \
   t = sc1 + (t >> DIGIT_BIT);             sc1 = t; sc2 += t >> DIGIT_BIT;     \
   } while (0);
\end{verbatim}
This computes a product and adds it to the ``secondary'' carry registers.

\begin{verbatim}
#define SQRADDDB                                                               \
   do { fp_word t;                                                             \
   t = ((fp_word)sc0) + ((fp_word)sc0) + c0; c0 = t;                                                 \
   t = ((fp_word)sc1) + ((fp_word)sc1) + c1 + (t >> DIGIT_BIT); c1 = t;                              \
   c2 = c2 + ((fp_word)sc2) + ((fp_word)sc2) + (t >> DIGIT_BIT);                                     \
   } while (0);
\end{verbatim}
This doubles the ``secondary'' carry registers and adds the sum to the main carry registers.  Really complicated.

\section{Montgomery with Comba}
Montgomery reduction is used in modular exponentiation and is most called function during
that operation.  It's important to make sure this routine is very fast or all is lost.

Unlike the two other comba routines this one does not use a single three--digit carry 
system.  It does have three--digit carries except that the routine steps through them
in the inner loop.  This means you cannot alias them to registers (at all).

To make matters simple though the three arrays of carries are stored in one array.  The 
``c0'' array resides in $c[0 \ldots OFF1-1]$, ``c1'' in $c[OFF1 \ldots OFF2-1]$ and ``c2'' in
$c[OFF2 \ldots OFF2+FP\_SIZE-1]$.  

\begin{verbatim}
#define MONT_START 
\end{verbatim}

This allows you to insert anything at the start that you need.

\begin{verbatim}
#define MONT_FINI
\end{verbatim}

This allows you to insert anything at the end that you need.

\begin{verbatim}
#define LOOP_START \
   mu = c[x] * mp;
\end{verbatim}

This computes the $\mu$ value for the inner loop.  You can safely alias $mu$ and $mp$ to
a register if you want.

\begin{verbatim}
#define INNERMUL                                      \
   do { fp_word t;                                    \
   _c[0] = t  = ((fp_word)_c[0] + (fp_word)cy) +      \
                (((fp_word)mu) * ((fp_word)*tmpm++)); \
   cy = (t >> DIGIT_BIT);                             \
   } while (0)
\end{verbatim}

This computes the inner product and adds it to the destination and carry variable $cy$.
This uses the $mu$ value computed above (can be in a register already) and the 
$cy$ which is a chaining carry.  Inside the INNERMUL loop the $cy$ value can be kept
inside a register (hint: it always starts as $cy = 0$ in the first iteration).

Upon completion of the inner loop the macro LOOP\_END is called which is used to fetch
$cy$ into the variable the C program can see.  This is where, if you cached $cy$ in a
register you would copy it to the locally accessible C variable.

\begin{verbatim}
#define PROPCARRY \
   do { fp_digit t = _c[0] += cy; cy = (t < cy); } while (0)
\end{verbatim}

This propagates the carry upwards by one digit.  

\documentclass[b5paper]{book}
\usepackage{hyperref}
\usepackage{makeidx}
\usepackage{amssymb}
\usepackage{color}
\usepackage{alltt}
\usepackage{graphicx}
\usepackage{layout}
\def\union{\cup}
\def\intersect{\cap}
\def\getsrandom{\stackrel{\rm R}{\gets}}
\def\cross{\times}
\def\cat{\hspace{0.5em} \| \hspace{0.5em}}
\def\catn{$\|$}
\def\divides{\hspace{0.3em} | \hspace{0.3em}}
\def\nequiv{\not\equiv}
\def\approx{\raisebox{0.2ex}{\mbox{\small $\sim$}}}
\def\lcm{{\rm lcm}}
\def\gcd{{\rm gcd}}
\def\log{{\rm log}}
\def\ord{{\rm ord}}
\def\abs{{\mathit abs}}
\def\rep{{\mathit rep}}
\def\mod{{\mathit\ mod\ }}
\renewcommand{\pmod}[1]{\ ({\rm mod\ }{#1})}
\newcommand{\floor}[1]{\left\lfloor{#1}\right\rfloor}
\newcommand{\ceil}[1]{\left\lceil{#1}\right\rceil}
\def\Or{{\rm\ or\ }}
\def\And{{\rm\ and\ }}
\def\iff{\hspace{1em}\Longleftrightarrow\hspace{1em}}
\def\implies{\Rightarrow}
\def\undefined{{\rm ``undefined"}}
\def\Proof{\vspace{1ex}\noindent {\bf Proof:}\hspace{1em}}
\let\oldphi\phi
\def\phi{\varphi}
\def\Pr{{\rm Pr}}
\newcommand{\str}[1]{{\mathbf{#1}}}
\def\F{{\mathbb F}}
\def\N{{\mathbb N}}
\def\Z{{\mathbb Z}}
\def\R{{\mathbb R}}
\def\C{{\mathbb C}}
\def\Q{{\mathbb Q}}
\definecolor{DGray}{gray}{0.5}
\newcommand{\emailaddr}[1]{\mbox{$<${#1}$>$}}
\def\twiddle{\raisebox{0.3ex}{\mbox{\tiny $\sim$}}}
\def\gap{\vspace{0.5ex}}
\makeindex
\begin{document}
\frontmatter
\pagestyle{empty}
\title{TomsFastMath User Manual \\ v0.13.0}
\author{Tom St Denis \\ tomstdenis@gmail.com}
\maketitle
This text and library are all hereby placed in the public domain.  This book has been formatted for B5 
[176x250] paper using the \LaTeX{} {\em book} macro package.

\vspace{13cm}

\begin{flushleft}This project was sponsored in part by  

Secure Science Corporation \url{http://www.securescience.net}.
\end{flushleft}

\tableofcontents
\listoffigures
\mainmatter
\pagestyle{headings}
\chapter{Introduction}
\section{What is TomsFastMath?}

TomsFastMath is meant to be a very fast yet still fairly portable and easy to port large
integer arithmetic library written in ISO C.  The goal specifically is to be able to perform
very fast modular exponentiations and other related functions required for ECC, DH and RSA
cryptosystems.

Most of the library is pure ISO C portable source code while a small portion (three files) contain
a mixture of ISO C and assembler inline fragments.  Compared to LibTomMath this new library is 
meant to be much faster while sacrificing flexibiltiy.  This is accomplished through several means.

\begin{enumerate}
   \item The new code is slightly messier and contains asm blocks.
   \item This uses fixed not multiple precision integers.  
   \item It is designed only for fast modular exponentiations [e.g. less flexibility].
\end{enumerate}

To mitigate some of the problems that arise from using assembler it has been carefully and 
appropriately used where it would make the most gain in performance.  Also we use macro's
for assembler code which allows new ports to be inserted easily.

The new code uses fixed precision arithmetic which means at compile time you choose a maximum 
precision and all numbers are limited to that.  This has the benefit of not requiring any
memory heap operations (which are slow) in any of the functions.  It has the downside that 
integers that are too large are truncated.

The goal of this library is to be able to perform modular exponentiations (with an odd modulus) very
fast.  This is what takes the most time in systems such as RSA and DH.  This also requires
fast multiplication and squaring and has the side effect of speeding up ECC operations as well.

\section{License}
TomsFastMath is public domain.

\section{Building}
To build the library simply type ``make''.  Or to install in typical *unix like directories use
``make install''.  Similarly a shared library can be built with ``make -f makefile.shared install''.

You can build the test program with ``make test''.  To perform simple static testing (useful to 
test out new assembly ports) use the stest program.  Type ``make stest'' and run it on your 
target.  The program will perform three multiplications, squarings and montgomery reductions.  
Likely if your assembly code is invalid this code will exhibit the bug.

\subsection{Intel CC}
In theory you should be able to build the library with

\begin{verbatim}
CFLAGS="-O3 -ip" CC=icc make IGNORE_SPEED=1
\end{verbatim}

However, Intels inline assembler is way less advanced than GCCs.  As a result it doesn't compile.  
Fortunately it doesn't really matter.

\subsection{MSVC}
The library doesn't build with MSVC.  Imagine that.

\subsection{Build Limitations}
TomsFastMath has the following build requirements which are non--portable but under most 
circumstances not problematic.

\begin{enumerate}
\item ``CHAR\_BIT'' must be eight.  
\item The ``fp\_digit'' type must be a multiple of eight bits long.
\item The ``fp\_word'' must be at least twice the length of fp\_digit.
\end{enumerate}

\subsection{Optimization Configuration}
By default TFM is configured for 32--bit digits using ISO C source code.  This mode while portable
is not very efficient.  While building the library (from scratch) you can define one of 
several ``CFLAGS'' defines.

For example, to build with with SSE2 optimizations type 

\begin{verbatim}
CFLAGS=-DTFM_SSE2 make clean libtfm.a
\end{verbatim}

\subsubsection{x86--32}  The ``x86--32'' mode is defined by ``TFM\_X86'' and covers all
i386 and beyond processors.  It requires GCC to build and only works with 32--bit digits.  In this 
mode fp\_digit is 32--bits and fp\_word is 64--bits.  This mode will be autodetected when building 
with GCC to an  ``i386'' target.  You can override this behaviour by defining TFM\_NO\_ASM or 
another optimization mode (such as SSE2).

\subsubsection{SSE2} The ``SSE2'' mode is defined by ``TFM\_SSE2'' and requires a Pentium 4, Pentium
M or Athlon64 processor.  It requires GCC to build.  Note that you shouldn't define both
TFM\_X86 and TFM\_SSE2 at the same time.   This mode only works with 32--bit digits.  In this 
mode fp\_digit is 32--bits and fp\_word is 64--bits.  While this mode will work on the AMD Athlon64 
series of processors it is less efficient than the native ``x86--64'' mode and not recommended.

There is an additional ``TFM\_PRESCOTT'' flag that you can define for P4 Prescott processors.  This causes
the mul/sqr functions to use x86\_32 and the montgomery reduction to use SSE2 which is (so far) the fastest
combination.  If you are using an older (e.g. Northwood) generation P4 don't define this.

\subsubsection{x86--64}  The ``x86--64'' mode is defined by ``TFM\_X86\_64'' and requires a 
``x86--64'' capable processor (Athlon64 and future Pentium processors).  It requires GCC to
build and only works with 64--bit digits.  Note that by enabling this mode it will automatically
enable 64--bit digits.  In this mode fp\_digit is 64--bits and fp\_word is 128--bits.  This mode will
be autodetected when building with GCC to an ``x86--64'' target.  You can override this behaviour by defining
TFM\_NO\_ASM.

\subsubsection{ARM}  The ``ARM'' mode is defined by ``TFM\_ARM'' and requires a ARMv4 with the M instructions (enhanced 
multipliers) or higher processor.  It requires GCC and works with 32--bit digits.  In this mode fp\_digit is 32--bits and 
fp\_word is 64--bits.

\subsubsection{PPC32} The ``PPC32'' mode is defined by ``TFM\_PPC32'' and requires a standard PPC processor.  It doesn't 
use altivec or other extensions so it should work on all compliant implementations of PPC.  It requires GCC and works
with 32--bit digits.  In this mode fp\_digit is 32--bits and fp\_word is 64--bits.

\subsubsection{PPC64} The ``PPC64'' mode is defined by ``TFM\_PPC64'' and requires a 64--bit PPC processor.  

\subsubsection{AVR32} The ``AVR32'' mode is defined by ``TFM\_AVR32'' and requires an Atmel AVR32 processor.  

\subsubsection{Future Releases}  Future releases will support additional platform optimizations.
Developers of MIPS and SPARC platforms are encouraged to submit GCC asm inline patches 
(see chapter \ref{chap:asmops} for more information).

\begin{figure}[here]
\begin{small}
\begin{center}
\begin{tabular}{|l|l|}
\hline \textbf{Processor} & \textbf{Recommended Mode} \\
\hline All 32--bit x86 platforms  & TFM\_X86 \\
\hline Pentium 4                  & TFM\_SSE2 \\
\hline Pentium 4 Prescott         & TFM\_SSE2 + TFM\_PRESCOTT \\
\hline Athlon64                   & TFM\_X86\_64 \\
\hline ARMv4 or higher with M     & TFM\_ARM \\
\hline G3/G4 (32-bit PPC)         & TFM\_PPC32 \\
\hline G5 (64-bit PPC)            & TFM\_PPC64 \\
\hline Atmel AVR32                & TFM\_AVR32 \\
\hline &\\
\hline x86--32 or x86--64 (with GCC) & Leave blank and let autodetect work \\
\hline
\end{tabular}
\caption{Recommended Build Modes}
\end{center}
\end{small}
\end{figure}

\subsection{Build Configurations}
TomsFastMath is configurable in terms of which unrolled code (if any) is included.  By default, the majority of the code is included which
results in large binaries.  The first flag to try out is TFM\_ALREADY\_SET which tells TFM to turn off \textbf{all} unrolled code.  This will
result in a smaller library but also a much slower library.

From this clean state, you can start enabling unrolled code for given cryptographic tasks at hand.  A series of TFM\_MULXYZ and TFM\_SQRXYZ macros
exist to enable specific unrolled code.  For instance, TFM\_MUL32 will enable a 32 digit unrolled multiplier.  For a complete list see the tfm.h header
file.  Keep in mind this is for digits not bits.  For example, you should enable TFM\_MUL16 if you are doing 1024-bit exptmods on a 64--bit platform, enable
TFM\_MUL32 on 32--bit platforms.

To help developers use ECC there are a set of defines for the five NIST curve sizes.  They are named TFM\_ECCXYZ where XYZ is one of 192, 224, 256, 384, or 521.  These
enable the multipliers and squaring code for a given curve, autodetecting 64--bit platforms as well.  

\subsection{Precision Configuration}
The precision of all integers in this library are fixed to a limited precision.  Essentially
the rule of setting the precision is if you plan on doing modular exponentiation with $k$--bit
numbers than the precision must be fixed to $2k$--bits plus four digits.  

This is changed by altering the value of ``FP\_MAX\_SIZE'' in tfm.h to your desired size.  By default, 
the library is configured to handle upto 2048--bit inputs to the modular exponentiator.  

\chapter{Getting Started}
\section{Data Types}
TomsFastMath is a large fixed precision integer library.  It provides the functionality to 
manipulate large signed integers through a relatively trivial api and a single data type.

The ``fp\_int'' or fixed precision integer is the data type that the functions operate with.  

\begin{verbatim}
typedef struct {
    fp_digit dp[FP_SIZE];
    int      used, 
             sign;
} fp_int;
\end{verbatim}

The \textbf{dp} member is the array of digits that forms the number.  It must always be zero 
padded.  The \textbf{used} member is the count of digits used in the array.  Although the 
precision is fixed the algorithms are still tuned to not process the entire array if it 
does not have to.  The \textbf{sign} indicates the sign of the integer.  It is \textbf{FP\_ZPOS} (0)
if the integer is zero or positive and \textbf{FP\_NEG} (1) otherwise.

\section{Initialization}
\subsection{Simple Initialization}
To initialize an integer to the default state of zero use the fp\_init() function.

\index{fp\_init}
\begin{verbatim}
void fp_init(fp_int *a);
\end{verbatim}

This will initialize the fp\_int $a$ to zero.  Note that the function fp\_zero() is an alias
for fp\_init().

\subsection{Initialize Small Constants}
To initialize an integer with a small single digit value use the fp\_set() function.

\index{fp\_set}
\begin{verbatim}
void fp_set(fp_int *a, fp_digit b);
\end{verbatim}

This will initialize $a$ and set it equal to the digit $b$.  

\subsection{Initialize Copy}
To initialize an integer with a copy of another integer use the fp\_init\_copy() function.

\index{fp\_init\_copy}
\begin{verbatim}
void fp_init_copy(fp_int *a, fp_int *b)
\end{verbatim}

This will initialize $a$ as a copy of $b$.  Note that for compatibility with LibTomMath the function
fp\_copy() is also provided.

\chapter{Arithmetic Operations}
\section{Odds and Evens}
To quickly and easily tell if an integer is zero, odd or even use the following functions.

\index{fp\_iszero} \index{fp\_iseven} \index{fp\_isodd}
\begin{verbatim}
int fp_iszero(fp_int *a);
int fp_iseven(fp_int *a);
int fp_isodd(fp_int *a);
\end{verbatim}

These will return \textbf{FP\_YES} if the answer to their respective questions is yes.  Otherwise they
return \textbf{FP\_NO}.  Note that these are implemented as macros and as such you should avoid using 
++ or --~-- operators on the input operand.

\section{Sign Manipulation}
To negate or compute the absolute of an integer use the following functions.

\index{fp\_neg} \index{fp\_abs}
\begin{verbatim}
void fp_neg(fp_int *a, fp_int *b);
void fp_abs(fp_int *a, fp_int *b);
\end{verbatim}
This will compute the negation (or absolute) of $a$ and store the result in $b$.  Note that these 
are implemented as macros and as such you should avoid using ++ or --~-- operators on the input 
operand.

\section{Comparisons}
To perform signed or unsigned comparisons use following functions.

\index{fp\_cmp} \index{fp\_cmp\_mag}
\begin{verbatim}
int fp_cmp(fp_int *a, fp_int *b);
int fp_cmp_mag(fp_int *a, fp_int *b);
\end{verbatim}
These will compare $a$ to $b$.  They will return \textbf{FP\_GT} if $a$ is larger than $b$, 
\textbf{FP\_EQ} if they are equal and \textbf{FP\_LT} if $a$ is less than $b$.

The function fp\_cmp performs signed comparisons while the other performs unsigned comparisons.

\section{Shifting}
To shift the digits of an fp\_int left or right use the following functions.

\index{fp\_lshd} \index{fp\_rshd}
\begin{verbatim}
void fp_lshd(fp_int *a, int x);
void fp_rshd(fp_int *a, int x);
\end{verbatim}

These will shift the digits of $a$ left (or right respectively) $x$ digits.  

To shift individual bits of an fp\_int use the following functions.

\index{fp\_div\_2d} \index{fp\_mod\_2d} \index{fp\_mul\_2d} \index{fp\_div\_2} \index{fp\_mul\_2}
\begin{verbatim}
void fp_div_2d(fp_int *a, int b, fp_int *c, fp_int *d);
void fp_mod_2d(fp_int *a, int b, fp_int *c);
void fp_mul_2d(fp_int *a, int b, fp_int *c);
void fp_mul_2(fp_int *a, fp_int *c);
void fp_div_2(fp_int *a, fp_int *c);
void fp_2expt(fp_int *a, int b);
\end{verbatim}
fp\_div\_2d() will divide $a$ by $2^b$ and store the quotient in $c$ and remainder in $d$.  Either of 
$c$ or $d$ can be \textbf{NULL} if their value is not required.  fp\_mod\_2d() is a shortcut to 
compute the remainder directly.  fp\_mul\_2d() will multiply $a$ by $2^b$ and store the result in $c$.  

The fp\_mul\_2() and fp\_div\_2() functions are optimized multiplication and divisions by two.  The 
function fp\_2expt() will compute $a = 2^b$ quickly.

To quickly count the number of least significant bits that are zero use the following function.

\index{fp\_cnt\_lsb}
\begin{verbatim}
int fp_cnt_lsb(fp_int *a);
\end{verbatim}
This will return the number of adjacent least significant bits that are zero.  This is equivalent 
to the number of times two evenly divides $a$.

\section{Basic Algebra}

The following functions round out the basic algebraic functionality of the library.

\index{fp\_add} \index{fp\_sub} \index{fp\_mul} \index{fp\_sqr} \index{fp\_div} \index{fp\_mod}
\begin{verbatim}
void fp_add(fp_int *a, fp_int *b, fp_int *c);
void fp_sub(fp_int *a, fp_int *b, fp_int *c);
void fp_mul(fp_int *a, fp_int *b, fp_int *c);
void fp_sqr(fp_int *a, fp_int *b);
int fp_div(fp_int *a, fp_int *b, fp_int *c, fp_int *d);
int fp_mod(fp_int *a, fp_int *b, fp_int *c);
\end{verbatim}

The functions fp\_add(), fp\_sub() and fp\_mul() perform their respective operations on $a$ and
$b$ and store the result in $c$.  The function fp\_sqr() computes $b = a^2$ and is faster than
using fp\_mul() to perform the same operation.

The function fp\_div() divides $a$ by $b$ and stores the quotient in $c$ and remainder in $d$.  Either 
of $c$ and $d$ can be \textbf{NULL} if the result is not required.  The function fp\_mod() is a simple 
shortcut to find the remainder.

\section{Modular Exponentiation}
To compute a modular exponentiation use the following function.

\index{fp\_exptmod}
\begin{verbatim}
int fp_exptmod(fp_int *a, fp_int *b, fp_int *c, fp_int *d);
\end{verbatim}
This computes $d \equiv a^b \mbox{ (mod }c\mbox{)}$ for any odd $c$ and $b$.  $b$ may be negative so long as 
$a^{-1} \mbox{ (mod }c\mbox{)}$ exists.  The initial value of $a$ may be larger than $c$.  The size of $c$ must be 
half of the maximum precision used during the build of the library.  For example, by default $c$ must be less 
than $2^{2048}$.  

\section{Number Theoretic}

To perform modular inverses, greatest common divisor or least common multiples use the following
functions.

\index{fp\_invmod} \index{fp\_gcd} \index{fp\_lcm}
\begin{verbatim}
int fp_invmod(fp_int *a, fp_int *b, fp_int *c);
void fp_gcd(fp_int *a, fp_int *b, fp_int *c);
void fp_lcm(fp_int *a, fp_int *b, fp_int *c);
\end{verbatim}

The fp\_invmod() function will find the modular inverse of $a$ modulo an odd modulus $b$ and store
it in $c$ (provided it exists).  The function fp\_gcd() will compute the greatest common
divisor of $a$ and $b$ and store it in $c$.  Similarly the fp\_lcm() function will compute
the least common multiple of $a$ and $b$ and store it in $c$.

\section{Prime Numbers}
To quickly test a number for primality call this function.

\index{fp\_isprime}
\index{fp\_isprime\_ex}
\begin{verbatim}
int fp_isprime_ex(fp_int *a, int t);
\end{verbatim}
This will return \textbf{FP\_YES} if $a$ is probably prime.  It uses 256 trial divisions and
$t$ rounds of Rabin-Miller testing.  Note that this routine performs modular exponentiations
which means that $a$ must be in a valid range of precision.
The valid range of $t$ is $1 \ldots 256$.

For backwards compatibility reasons the API function fp\_isprime(a) is still available
and simply calls fp\_isprime\_ex(a, 8).

\chapter{Porting TomsFastMath}
\label{chap:asmops}
\section{Getting Started}
Porting TomsFastMath to a given processor target is usually a simple procedure.  For the most part 
assembly is used to get around the lack of a ``add with carry'' operation in the C language.  To
make matters simpler the use of assembler is through macro blocks.

Each ``port'' is defined by a block of code that re-defines the portable ISO C macros with assembler
inline blocks.  To add a new port you must designate a TFM\_XXX define that will enable your 
port when built.

\section{Multiply with Comba}
The file ``fp\_mul\_comba.c'' is responsible for providing the fast multiplication within the 
library.  This comba multiplication is fairly simple.  It uses a sliding three digit carry 
system with the variables $c0$, $c1$, $c2$.  For every digit of output $c0$ is the what will
be that digit, $c1$ will carry into the next digit and $c2$ will be the ``c1'' carry for
the next digit.  For every ``next'' digit effectively $c0$ is stored as output, $c1$ moves into
$c0$, $c2$ into $c1$ and zero into $c2$.

The following macros define the assmebler interface to the code.

\begin{verbatim}
#define COMBA_START 
\end{verbatim}

This is issued at the beginning of the multiplication function.  This is in place to allow you to
initialize any registers or machine words required.  You can leave it blank if you do not need 
it.

\begin{verbatim}
#define COMBA_CLEAR \
   c0 = c1 = c2 = 0;
\end{verbatim}

This clears the three comba carries.  If you are going to place carries in registers then 
zero the appropriate registers.  Note that the functions do not use $c0$, $c1$ or $c2$ directly
so you are free to ignore these varibles and use registers directly.

\begin{verbatim}
#define COMBA_FORWARD \
   c0 = c1; c1 = c2; c2 = 0;
\end{verbatim}

This propagates the carries after a digit has been produced.  

\begin{verbatim}
#define COMBA_STORE(x) \
   x = c0;
\end{verbatim}

This stores the $c0$ digit in the memory location specified by $x$.  Note that if you manually 
aliased $c0$ with a register than just store that register in $x$.  

\begin{verbatim}
#define COMBA_STORE2(x) \
   x = c1;
\end{verbatim}

This stores the $c1$ digit in the memory location specified by $x$.  Note that if you manually 
aliased $c1$ with a register than just store that register in $x$.  

\begin{verbatim}
#define COMBA_FINI
\end{verbatim}

If at the end of the function you need to perform some action fill this macro in. 

\begin{verbatim}
#define MULADD(i, j)                                          \
   t  = ((fp_word)i) * ((fp_word)j);                          \
   c0 = (c0 + t);              if (c0 < ((fp_digit)t))  ++c1; \
   c1 = (c1 + (t>>DIGIT_BIT)); if (c1 < (t>>DIGIT_BIT)) ++c2;
\end{verbatim}

This macro performs the ``multiply and add'' step that is central to the comba
multiplier.  It multiplies the fp\_digits $i$ and $j$ to produce a fp\_word result.  Effectively
the double--digit value is added to the three-digit carry formed by $c0$, $c1$, $c2$ where $c0$
is the least significant digit.

\section{Squaring with Comba}
Squaring is similar to multiplication except that it uses a special ``multiply and add twice'' macro
that replaces multiplications that are not required.

\begin{verbatim}
#define COMBA_START
\end{verbatim}

This allows for any initialization code you might have.  

\begin{verbatim}
#define CLEAR_CARRY \
   c0 = c1 = c2 = 0;
\end{verbatim}

This will clear the carries.  Like multiplication you can safely alias the three carry variables
to registers if you can/want to.

\begin{verbatim}
#define COMBA_STORE(x) \
   x = c0;
\end{verbatim}

Store the $c0$ carry to a given memory location.

\begin{verbatim}
#define COMBA_STORE2(x) \
   x = c1;
\end{verbatim}

Store the $c1$ carry to a given memory location.

\begin{verbatim}
#define CARRY_FORWARD \
   c0 = c1; c1 = c2; c2 = 0;
\end{verbatim}

Forward propagate all three carry variables.

\begin{verbatim}
#define COMBA_FINI
\end{verbatim}

If you need to clean up at the end of the function.

\begin{verbatim}
/* multiplies point i and j, updates carry "c1" and digit c2 */
#define SQRADD(i, j)                       \
   t  = ((fp_word)i) * ((fp_word)j);       \
   c0 = (c0 + t);              if (c0 < ((fp_digit)t))  ++c1; \
   c1 = (c1 + (t>>DIGIT_BIT)); if (c1 < (t>>DIGIT_BIT)) ++c2; 
\end{verbatim}

This is essentially the MULADD macro from the multiplication code.

\begin{verbatim}
/* for squaring some of the terms are doubled... */
#define SQRADD2(i, j)                       \
   t  = ((fp_word)i) * ((fp_word)j);       \
   c0 = (c0 + t);              if (c0 < ((fp_digit)t))  ++c1; \
   c1 = (c1 + (t>>DIGIT_BIT)); if (c1 < (t>>DIGIT_BIT)) ++c2; \
   c0 = (c0 + t);              if (c0 < ((fp_digit)t))  ++c1; \
   c1 = (c1 + (t>>DIGIT_BIT)); if (c1 < (t>>DIGIT_BIT)) ++c2; 
\end{verbatim}

This is like SQRADD except it adds the produce twice.  It's similar to 
computing SQRADD(i, j*2).

To further make things interesting the squaring code also has ``doubles'' (see my LTM book chapter five...) which are
handled with these macros.

\begin{verbatim}
#define SQRADDSC(i, j)                                                         \
   do { fp_word t;                                                             \
      t =  ((fp_word)i) * ((fp_word)j);                                        \
      sc0 = (fp_digit)t; sc1 = (t >> DIGIT_BIT); sc2 = 0;                      \
   } while (0);
\end{verbatim}
This computes a product and stores it in the ``secondary'' carry registers $\left < sc0, sc1, sc2 \right >$.

\begin{verbatim}
#define SQRADDAC(i, j)                                                         \
   do { fp_word t;                                                             \
   t = sc0 + ((fp_word)i) * ((fp_word)j);  sc0 = t;                            \
   t = sc1 + (t >> DIGIT_BIT);             sc1 = t; sc2 += t >> DIGIT_BIT;     \
   } while (0);
\end{verbatim}
This computes a product and adds it to the ``secondary'' carry registers.

\begin{verbatim}
#define SQRADDDB                                                               \
   do { fp_word t;                                                             \
   t = ((fp_word)sc0) + ((fp_word)sc0) + c0; c0 = t;                                                 \
   t = ((fp_word)sc1) + ((fp_word)sc1) + c1 + (t >> DIGIT_BIT); c1 = t;                              \
   c2 = c2 + ((fp_word)sc2) + ((fp_word)sc2) + (t >> DIGIT_BIT);                                     \
   } while (0);
\end{verbatim}
This doubles the ``secondary'' carry registers and adds the sum to the main carry registers.  Really complicated.

\section{Montgomery with Comba}
Montgomery reduction is used in modular exponentiation and is most called function during
that operation.  It's important to make sure this routine is very fast or all is lost.

Unlike the two other comba routines this one does not use a single three--digit carry 
system.  It does have three--digit carries except that the routine steps through them
in the inner loop.  This means you cannot alias them to registers (at all).

To make matters simple though the three arrays of carries are stored in one array.  The 
``c0'' array resides in $c[0 \ldots OFF1-1]$, ``c1'' in $c[OFF1 \ldots OFF2-1]$ and ``c2'' in
$c[OFF2 \ldots OFF2+FP\_SIZE-1]$.  

\begin{verbatim}
#define MONT_START 
\end{verbatim}

This allows you to insert anything at the start that you need.

\begin{verbatim}
#define MONT_FINI
\end{verbatim}

This allows you to insert anything at the end that you need.

\begin{verbatim}
#define LOOP_START \
   mu = c[x] * mp;
\end{verbatim}

This computes the $\mu$ value for the inner loop.  You can safely alias $mu$ and $mp$ to
a register if you want.

\begin{verbatim}
#define INNERMUL                                      \
   do { fp_word t;                                    \
   _c[0] = t  = ((fp_word)_c[0] + (fp_word)cy) +      \
                (((fp_word)mu) * ((fp_word)*tmpm++)); \
   cy = (t >> DIGIT_BIT);                             \
   } while (0)
\end{verbatim}

This computes the inner product and adds it to the destination and carry variable $cy$.
This uses the $mu$ value computed above (can be in a register already) and the 
$cy$ which is a chaining carry.  Inside the INNERMUL loop the $cy$ value can be kept
inside a register (hint: it always starts as $cy = 0$ in the first iteration).

Upon completion of the inner loop the macro LOOP\_END is called which is used to fetch
$cy$ into the variable the C program can see.  This is where, if you cached $cy$ in a
register you would copy it to the locally accessible C variable.

\begin{verbatim}
#define PROPCARRY \
   do { fp_digit t = _c[0] += cy; cy = (t < cy); } while (0)
\end{verbatim}

This propagates the carry upwards by one digit.  

\documentclass[b5paper]{book}
\usepackage{hyperref}
\usepackage{makeidx}
\usepackage{amssymb}
\usepackage{color}
\usepackage{alltt}
\usepackage{graphicx}
\usepackage{layout}
\def\union{\cup}
\def\intersect{\cap}
\def\getsrandom{\stackrel{\rm R}{\gets}}
\def\cross{\times}
\def\cat{\hspace{0.5em} \| \hspace{0.5em}}
\def\catn{$\|$}
\def\divides{\hspace{0.3em} | \hspace{0.3em}}
\def\nequiv{\not\equiv}
\def\approx{\raisebox{0.2ex}{\mbox{\small $\sim$}}}
\def\lcm{{\rm lcm}}
\def\gcd{{\rm gcd}}
\def\log{{\rm log}}
\def\ord{{\rm ord}}
\def\abs{{\mathit abs}}
\def\rep{{\mathit rep}}
\def\mod{{\mathit\ mod\ }}
\renewcommand{\pmod}[1]{\ ({\rm mod\ }{#1})}
\newcommand{\floor}[1]{\left\lfloor{#1}\right\rfloor}
\newcommand{\ceil}[1]{\left\lceil{#1}\right\rceil}
\def\Or{{\rm\ or\ }}
\def\And{{\rm\ and\ }}
\def\iff{\hspace{1em}\Longleftrightarrow\hspace{1em}}
\def\implies{\Rightarrow}
\def\undefined{{\rm ``undefined"}}
\def\Proof{\vspace{1ex}\noindent {\bf Proof:}\hspace{1em}}
\let\oldphi\phi
\def\phi{\varphi}
\def\Pr{{\rm Pr}}
\newcommand{\str}[1]{{\mathbf{#1}}}
\def\F{{\mathbb F}}
\def\N{{\mathbb N}}
\def\Z{{\mathbb Z}}
\def\R{{\mathbb R}}
\def\C{{\mathbb C}}
\def\Q{{\mathbb Q}}
\definecolor{DGray}{gray}{0.5}
\newcommand{\emailaddr}[1]{\mbox{$<${#1}$>$}}
\def\twiddle{\raisebox{0.3ex}{\mbox{\tiny $\sim$}}}
\def\gap{\vspace{0.5ex}}
\makeindex
\begin{document}
\frontmatter
\pagestyle{empty}
\title{TomsFastMath User Manual \\ v0.13.0}
\author{Tom St Denis \\ tomstdenis@gmail.com}
\maketitle
This text and library are all hereby placed in the public domain.  This book has been formatted for B5 
[176x250] paper using the \LaTeX{} {\em book} macro package.

\vspace{13cm}

\begin{flushleft}This project was sponsored in part by  

Secure Science Corporation \url{http://www.securescience.net}.
\end{flushleft}

\tableofcontents
\listoffigures
\mainmatter
\pagestyle{headings}
\chapter{Introduction}
\section{What is TomsFastMath?}

TomsFastMath is meant to be a very fast yet still fairly portable and easy to port large
integer arithmetic library written in ISO C.  The goal specifically is to be able to perform
very fast modular exponentiations and other related functions required for ECC, DH and RSA
cryptosystems.

Most of the library is pure ISO C portable source code while a small portion (three files) contain
a mixture of ISO C and assembler inline fragments.  Compared to LibTomMath this new library is 
meant to be much faster while sacrificing flexibiltiy.  This is accomplished through several means.

\begin{enumerate}
   \item The new code is slightly messier and contains asm blocks.
   \item This uses fixed not multiple precision integers.  
   \item It is designed only for fast modular exponentiations [e.g. less flexibility].
\end{enumerate}

To mitigate some of the problems that arise from using assembler it has been carefully and 
appropriately used where it would make the most gain in performance.  Also we use macro's
for assembler code which allows new ports to be inserted easily.

The new code uses fixed precision arithmetic which means at compile time you choose a maximum 
precision and all numbers are limited to that.  This has the benefit of not requiring any
memory heap operations (which are slow) in any of the functions.  It has the downside that 
integers that are too large are truncated.

The goal of this library is to be able to perform modular exponentiations (with an odd modulus) very
fast.  This is what takes the most time in systems such as RSA and DH.  This also requires
fast multiplication and squaring and has the side effect of speeding up ECC operations as well.

\section{License}
TomsFastMath is public domain.

\section{Building}
To build the library simply type ``make''.  Or to install in typical *unix like directories use
``make install''.  Similarly a shared library can be built with ``make -f makefile.shared install''.

You can build the test program with ``make test''.  To perform simple static testing (useful to 
test out new assembly ports) use the stest program.  Type ``make stest'' and run it on your 
target.  The program will perform three multiplications, squarings and montgomery reductions.  
Likely if your assembly code is invalid this code will exhibit the bug.

\subsection{Intel CC}
In theory you should be able to build the library with

\begin{verbatim}
CFLAGS="-O3 -ip" CC=icc make IGNORE_SPEED=1
\end{verbatim}

However, Intels inline assembler is way less advanced than GCCs.  As a result it doesn't compile.  
Fortunately it doesn't really matter.

\subsection{MSVC}
The library doesn't build with MSVC.  Imagine that.

\subsection{Build Limitations}
TomsFastMath has the following build requirements which are non--portable but under most 
circumstances not problematic.

\begin{enumerate}
\item ``CHAR\_BIT'' must be eight.  
\item The ``fp\_digit'' type must be a multiple of eight bits long.
\item The ``fp\_word'' must be at least twice the length of fp\_digit.
\end{enumerate}

\subsection{Optimization Configuration}
By default TFM is configured for 32--bit digits using ISO C source code.  This mode while portable
is not very efficient.  While building the library (from scratch) you can define one of 
several ``CFLAGS'' defines.

For example, to build with with SSE2 optimizations type 

\begin{verbatim}
CFLAGS=-DTFM_SSE2 make clean libtfm.a
\end{verbatim}

\subsubsection{x86--32}  The ``x86--32'' mode is defined by ``TFM\_X86'' and covers all
i386 and beyond processors.  It requires GCC to build and only works with 32--bit digits.  In this 
mode fp\_digit is 32--bits and fp\_word is 64--bits.  This mode will be autodetected when building 
with GCC to an  ``i386'' target.  You can override this behaviour by defining TFM\_NO\_ASM or 
another optimization mode (such as SSE2).

\subsubsection{SSE2} The ``SSE2'' mode is defined by ``TFM\_SSE2'' and requires a Pentium 4, Pentium
M or Athlon64 processor.  It requires GCC to build.  Note that you shouldn't define both
TFM\_X86 and TFM\_SSE2 at the same time.   This mode only works with 32--bit digits.  In this 
mode fp\_digit is 32--bits and fp\_word is 64--bits.  While this mode will work on the AMD Athlon64 
series of processors it is less efficient than the native ``x86--64'' mode and not recommended.

There is an additional ``TFM\_PRESCOTT'' flag that you can define for P4 Prescott processors.  This causes
the mul/sqr functions to use x86\_32 and the montgomery reduction to use SSE2 which is (so far) the fastest
combination.  If you are using an older (e.g. Northwood) generation P4 don't define this.

\subsubsection{x86--64}  The ``x86--64'' mode is defined by ``TFM\_X86\_64'' and requires a 
``x86--64'' capable processor (Athlon64 and future Pentium processors).  It requires GCC to
build and only works with 64--bit digits.  Note that by enabling this mode it will automatically
enable 64--bit digits.  In this mode fp\_digit is 64--bits and fp\_word is 128--bits.  This mode will
be autodetected when building with GCC to an ``x86--64'' target.  You can override this behaviour by defining
TFM\_NO\_ASM.

\subsubsection{ARM}  The ``ARM'' mode is defined by ``TFM\_ARM'' and requires a ARMv4 with the M instructions (enhanced 
multipliers) or higher processor.  It requires GCC and works with 32--bit digits.  In this mode fp\_digit is 32--bits and 
fp\_word is 64--bits.

\subsubsection{PPC32} The ``PPC32'' mode is defined by ``TFM\_PPC32'' and requires a standard PPC processor.  It doesn't 
use altivec or other extensions so it should work on all compliant implementations of PPC.  It requires GCC and works
with 32--bit digits.  In this mode fp\_digit is 32--bits and fp\_word is 64--bits.

\subsubsection{PPC64} The ``PPC64'' mode is defined by ``TFM\_PPC64'' and requires a 64--bit PPC processor.  

\subsubsection{AVR32} The ``AVR32'' mode is defined by ``TFM\_AVR32'' and requires an Atmel AVR32 processor.  

\subsubsection{Future Releases}  Future releases will support additional platform optimizations.
Developers of MIPS and SPARC platforms are encouraged to submit GCC asm inline patches 
(see chapter \ref{chap:asmops} for more information).

\begin{figure}[here]
\begin{small}
\begin{center}
\begin{tabular}{|l|l|}
\hline \textbf{Processor} & \textbf{Recommended Mode} \\
\hline All 32--bit x86 platforms  & TFM\_X86 \\
\hline Pentium 4                  & TFM\_SSE2 \\
\hline Pentium 4 Prescott         & TFM\_SSE2 + TFM\_PRESCOTT \\
\hline Athlon64                   & TFM\_X86\_64 \\
\hline ARMv4 or higher with M     & TFM\_ARM \\
\hline G3/G4 (32-bit PPC)         & TFM\_PPC32 \\
\hline G5 (64-bit PPC)            & TFM\_PPC64 \\
\hline Atmel AVR32                & TFM\_AVR32 \\
\hline &\\
\hline x86--32 or x86--64 (with GCC) & Leave blank and let autodetect work \\
\hline
\end{tabular}
\caption{Recommended Build Modes}
\end{center}
\end{small}
\end{figure}

\subsection{Build Configurations}
TomsFastMath is configurable in terms of which unrolled code (if any) is included.  By default, the majority of the code is included which
results in large binaries.  The first flag to try out is TFM\_ALREADY\_SET which tells TFM to turn off \textbf{all} unrolled code.  This will
result in a smaller library but also a much slower library.

From this clean state, you can start enabling unrolled code for given cryptographic tasks at hand.  A series of TFM\_MULXYZ and TFM\_SQRXYZ macros
exist to enable specific unrolled code.  For instance, TFM\_MUL32 will enable a 32 digit unrolled multiplier.  For a complete list see the tfm.h header
file.  Keep in mind this is for digits not bits.  For example, you should enable TFM\_MUL16 if you are doing 1024-bit exptmods on a 64--bit platform, enable
TFM\_MUL32 on 32--bit platforms.

To help developers use ECC there are a set of defines for the five NIST curve sizes.  They are named TFM\_ECCXYZ where XYZ is one of 192, 224, 256, 384, or 521.  These
enable the multipliers and squaring code for a given curve, autodetecting 64--bit platforms as well.  

\subsection{Precision Configuration}
The precision of all integers in this library are fixed to a limited precision.  Essentially
the rule of setting the precision is if you plan on doing modular exponentiation with $k$--bit
numbers than the precision must be fixed to $2k$--bits plus four digits.  

This is changed by altering the value of ``FP\_MAX\_SIZE'' in tfm.h to your desired size.  By default, 
the library is configured to handle upto 2048--bit inputs to the modular exponentiator.  

\chapter{Getting Started}
\section{Data Types}
TomsFastMath is a large fixed precision integer library.  It provides the functionality to 
manipulate large signed integers through a relatively trivial api and a single data type.

The ``fp\_int'' or fixed precision integer is the data type that the functions operate with.  

\begin{verbatim}
typedef struct {
    fp_digit dp[FP_SIZE];
    int      used, 
             sign;
} fp_int;
\end{verbatim}

The \textbf{dp} member is the array of digits that forms the number.  It must always be zero 
padded.  The \textbf{used} member is the count of digits used in the array.  Although the 
precision is fixed the algorithms are still tuned to not process the entire array if it 
does not have to.  The \textbf{sign} indicates the sign of the integer.  It is \textbf{FP\_ZPOS} (0)
if the integer is zero or positive and \textbf{FP\_NEG} (1) otherwise.

\section{Initialization}
\subsection{Simple Initialization}
To initialize an integer to the default state of zero use the fp\_init() function.

\index{fp\_init}
\begin{verbatim}
void fp_init(fp_int *a);
\end{verbatim}

This will initialize the fp\_int $a$ to zero.  Note that the function fp\_zero() is an alias
for fp\_init().

\subsection{Initialize Small Constants}
To initialize an integer with a small single digit value use the fp\_set() function.

\index{fp\_set}
\begin{verbatim}
void fp_set(fp_int *a, fp_digit b);
\end{verbatim}

This will initialize $a$ and set it equal to the digit $b$.  

\subsection{Initialize Copy}
To initialize an integer with a copy of another integer use the fp\_init\_copy() function.

\index{fp\_init\_copy}
\begin{verbatim}
void fp_init_copy(fp_int *a, fp_int *b)
\end{verbatim}

This will initialize $a$ as a copy of $b$.  Note that for compatibility with LibTomMath the function
fp\_copy() is also provided.

\chapter{Arithmetic Operations}
\section{Odds and Evens}
To quickly and easily tell if an integer is zero, odd or even use the following functions.

\index{fp\_iszero} \index{fp\_iseven} \index{fp\_isodd}
\begin{verbatim}
int fp_iszero(fp_int *a);
int fp_iseven(fp_int *a);
int fp_isodd(fp_int *a);
\end{verbatim}

These will return \textbf{FP\_YES} if the answer to their respective questions is yes.  Otherwise they
return \textbf{FP\_NO}.  Note that these are implemented as macros and as such you should avoid using 
++ or --~-- operators on the input operand.

\section{Sign Manipulation}
To negate or compute the absolute of an integer use the following functions.

\index{fp\_neg} \index{fp\_abs}
\begin{verbatim}
void fp_neg(fp_int *a, fp_int *b);
void fp_abs(fp_int *a, fp_int *b);
\end{verbatim}
This will compute the negation (or absolute) of $a$ and store the result in $b$.  Note that these 
are implemented as macros and as such you should avoid using ++ or --~-- operators on the input 
operand.

\section{Comparisons}
To perform signed or unsigned comparisons use following functions.

\index{fp\_cmp} \index{fp\_cmp\_mag}
\begin{verbatim}
int fp_cmp(fp_int *a, fp_int *b);
int fp_cmp_mag(fp_int *a, fp_int *b);
\end{verbatim}
These will compare $a$ to $b$.  They will return \textbf{FP\_GT} if $a$ is larger than $b$, 
\textbf{FP\_EQ} if they are equal and \textbf{FP\_LT} if $a$ is less than $b$.

The function fp\_cmp performs signed comparisons while the other performs unsigned comparisons.

\section{Shifting}
To shift the digits of an fp\_int left or right use the following functions.

\index{fp\_lshd} \index{fp\_rshd}
\begin{verbatim}
void fp_lshd(fp_int *a, int x);
void fp_rshd(fp_int *a, int x);
\end{verbatim}

These will shift the digits of $a$ left (or right respectively) $x$ digits.  

To shift individual bits of an fp\_int use the following functions.

\index{fp\_div\_2d} \index{fp\_mod\_2d} \index{fp\_mul\_2d} \index{fp\_div\_2} \index{fp\_mul\_2}
\begin{verbatim}
void fp_div_2d(fp_int *a, int b, fp_int *c, fp_int *d);
void fp_mod_2d(fp_int *a, int b, fp_int *c);
void fp_mul_2d(fp_int *a, int b, fp_int *c);
void fp_mul_2(fp_int *a, fp_int *c);
void fp_div_2(fp_int *a, fp_int *c);
void fp_2expt(fp_int *a, int b);
\end{verbatim}
fp\_div\_2d() will divide $a$ by $2^b$ and store the quotient in $c$ and remainder in $d$.  Either of 
$c$ or $d$ can be \textbf{NULL} if their value is not required.  fp\_mod\_2d() is a shortcut to 
compute the remainder directly.  fp\_mul\_2d() will multiply $a$ by $2^b$ and store the result in $c$.  

The fp\_mul\_2() and fp\_div\_2() functions are optimized multiplication and divisions by two.  The 
function fp\_2expt() will compute $a = 2^b$ quickly.

To quickly count the number of least significant bits that are zero use the following function.

\index{fp\_cnt\_lsb}
\begin{verbatim}
int fp_cnt_lsb(fp_int *a);
\end{verbatim}
This will return the number of adjacent least significant bits that are zero.  This is equivalent 
to the number of times two evenly divides $a$.

\section{Basic Algebra}

The following functions round out the basic algebraic functionality of the library.

\index{fp\_add} \index{fp\_sub} \index{fp\_mul} \index{fp\_sqr} \index{fp\_div} \index{fp\_mod}
\begin{verbatim}
void fp_add(fp_int *a, fp_int *b, fp_int *c);
void fp_sub(fp_int *a, fp_int *b, fp_int *c);
void fp_mul(fp_int *a, fp_int *b, fp_int *c);
void fp_sqr(fp_int *a, fp_int *b);
int fp_div(fp_int *a, fp_int *b, fp_int *c, fp_int *d);
int fp_mod(fp_int *a, fp_int *b, fp_int *c);
\end{verbatim}

The functions fp\_add(), fp\_sub() and fp\_mul() perform their respective operations on $a$ and
$b$ and store the result in $c$.  The function fp\_sqr() computes $b = a^2$ and is faster than
using fp\_mul() to perform the same operation.

The function fp\_div() divides $a$ by $b$ and stores the quotient in $c$ and remainder in $d$.  Either 
of $c$ and $d$ can be \textbf{NULL} if the result is not required.  The function fp\_mod() is a simple 
shortcut to find the remainder.

\section{Modular Exponentiation}
To compute a modular exponentiation use the following function.

\index{fp\_exptmod}
\begin{verbatim}
int fp_exptmod(fp_int *a, fp_int *b, fp_int *c, fp_int *d);
\end{verbatim}
This computes $d \equiv a^b \mbox{ (mod }c\mbox{)}$ for any odd $c$ and $b$.  $b$ may be negative so long as 
$a^{-1} \mbox{ (mod }c\mbox{)}$ exists.  The initial value of $a$ may be larger than $c$.  The size of $c$ must be 
half of the maximum precision used during the build of the library.  For example, by default $c$ must be less 
than $2^{2048}$.  

\section{Number Theoretic}

To perform modular inverses, greatest common divisor or least common multiples use the following
functions.

\index{fp\_invmod} \index{fp\_gcd} \index{fp\_lcm}
\begin{verbatim}
int fp_invmod(fp_int *a, fp_int *b, fp_int *c);
void fp_gcd(fp_int *a, fp_int *b, fp_int *c);
void fp_lcm(fp_int *a, fp_int *b, fp_int *c);
\end{verbatim}

The fp\_invmod() function will find the modular inverse of $a$ modulo an odd modulus $b$ and store
it in $c$ (provided it exists).  The function fp\_gcd() will compute the greatest common
divisor of $a$ and $b$ and store it in $c$.  Similarly the fp\_lcm() function will compute
the least common multiple of $a$ and $b$ and store it in $c$.

\section{Prime Numbers}
To quickly test a number for primality call this function.

\index{fp\_isprime}
\index{fp\_isprime\_ex}
\begin{verbatim}
int fp_isprime_ex(fp_int *a, int t);
\end{verbatim}
This will return \textbf{FP\_YES} if $a$ is probably prime.  It uses 256 trial divisions and
$t$ rounds of Rabin-Miller testing.  Note that this routine performs modular exponentiations
which means that $a$ must be in a valid range of precision.
The valid range of $t$ is $1 \ldots 256$.

For backwards compatibility reasons the API function fp\_isprime(a) is still available
and simply calls fp\_isprime\_ex(a, 8).

\chapter{Porting TomsFastMath}
\label{chap:asmops}
\section{Getting Started}
Porting TomsFastMath to a given processor target is usually a simple procedure.  For the most part 
assembly is used to get around the lack of a ``add with carry'' operation in the C language.  To
make matters simpler the use of assembler is through macro blocks.

Each ``port'' is defined by a block of code that re-defines the portable ISO C macros with assembler
inline blocks.  To add a new port you must designate a TFM\_XXX define that will enable your 
port when built.

\section{Multiply with Comba}
The file ``fp\_mul\_comba.c'' is responsible for providing the fast multiplication within the 
library.  This comba multiplication is fairly simple.  It uses a sliding three digit carry 
system with the variables $c0$, $c1$, $c2$.  For every digit of output $c0$ is the what will
be that digit, $c1$ will carry into the next digit and $c2$ will be the ``c1'' carry for
the next digit.  For every ``next'' digit effectively $c0$ is stored as output, $c1$ moves into
$c0$, $c2$ into $c1$ and zero into $c2$.

The following macros define the assmebler interface to the code.

\begin{verbatim}
#define COMBA_START 
\end{verbatim}

This is issued at the beginning of the multiplication function.  This is in place to allow you to
initialize any registers or machine words required.  You can leave it blank if you do not need 
it.

\begin{verbatim}
#define COMBA_CLEAR \
   c0 = c1 = c2 = 0;
\end{verbatim}

This clears the three comba carries.  If you are going to place carries in registers then 
zero the appropriate registers.  Note that the functions do not use $c0$, $c1$ or $c2$ directly
so you are free to ignore these varibles and use registers directly.

\begin{verbatim}
#define COMBA_FORWARD \
   c0 = c1; c1 = c2; c2 = 0;
\end{verbatim}

This propagates the carries after a digit has been produced.  

\begin{verbatim}
#define COMBA_STORE(x) \
   x = c0;
\end{verbatim}

This stores the $c0$ digit in the memory location specified by $x$.  Note that if you manually 
aliased $c0$ with a register than just store that register in $x$.  

\begin{verbatim}
#define COMBA_STORE2(x) \
   x = c1;
\end{verbatim}

This stores the $c1$ digit in the memory location specified by $x$.  Note that if you manually 
aliased $c1$ with a register than just store that register in $x$.  

\begin{verbatim}
#define COMBA_FINI
\end{verbatim}

If at the end of the function you need to perform some action fill this macro in. 

\begin{verbatim}
#define MULADD(i, j)                                          \
   t  = ((fp_word)i) * ((fp_word)j);                          \
   c0 = (c0 + t);              if (c0 < ((fp_digit)t))  ++c1; \
   c1 = (c1 + (t>>DIGIT_BIT)); if (c1 < (t>>DIGIT_BIT)) ++c2;
\end{verbatim}

This macro performs the ``multiply and add'' step that is central to the comba
multiplier.  It multiplies the fp\_digits $i$ and $j$ to produce a fp\_word result.  Effectively
the double--digit value is added to the three-digit carry formed by $c0$, $c1$, $c2$ where $c0$
is the least significant digit.

\section{Squaring with Comba}
Squaring is similar to multiplication except that it uses a special ``multiply and add twice'' macro
that replaces multiplications that are not required.

\begin{verbatim}
#define COMBA_START
\end{verbatim}

This allows for any initialization code you might have.  

\begin{verbatim}
#define CLEAR_CARRY \
   c0 = c1 = c2 = 0;
\end{verbatim}

This will clear the carries.  Like multiplication you can safely alias the three carry variables
to registers if you can/want to.

\begin{verbatim}
#define COMBA_STORE(x) \
   x = c0;
\end{verbatim}

Store the $c0$ carry to a given memory location.

\begin{verbatim}
#define COMBA_STORE2(x) \
   x = c1;
\end{verbatim}

Store the $c1$ carry to a given memory location.

\begin{verbatim}
#define CARRY_FORWARD \
   c0 = c1; c1 = c2; c2 = 0;
\end{verbatim}

Forward propagate all three carry variables.

\begin{verbatim}
#define COMBA_FINI
\end{verbatim}

If you need to clean up at the end of the function.

\begin{verbatim}
/* multiplies point i and j, updates carry "c1" and digit c2 */
#define SQRADD(i, j)                       \
   t  = ((fp_word)i) * ((fp_word)j);       \
   c0 = (c0 + t);              if (c0 < ((fp_digit)t))  ++c1; \
   c1 = (c1 + (t>>DIGIT_BIT)); if (c1 < (t>>DIGIT_BIT)) ++c2; 
\end{verbatim}

This is essentially the MULADD macro from the multiplication code.

\begin{verbatim}
/* for squaring some of the terms are doubled... */
#define SQRADD2(i, j)                       \
   t  = ((fp_word)i) * ((fp_word)j);       \
   c0 = (c0 + t);              if (c0 < ((fp_digit)t))  ++c1; \
   c1 = (c1 + (t>>DIGIT_BIT)); if (c1 < (t>>DIGIT_BIT)) ++c2; \
   c0 = (c0 + t);              if (c0 < ((fp_digit)t))  ++c1; \
   c1 = (c1 + (t>>DIGIT_BIT)); if (c1 < (t>>DIGIT_BIT)) ++c2; 
\end{verbatim}

This is like SQRADD except it adds the produce twice.  It's similar to 
computing SQRADD(i, j*2).

To further make things interesting the squaring code also has ``doubles'' (see my LTM book chapter five...) which are
handled with these macros.

\begin{verbatim}
#define SQRADDSC(i, j)                                                         \
   do { fp_word t;                                                             \
      t =  ((fp_word)i) * ((fp_word)j);                                        \
      sc0 = (fp_digit)t; sc1 = (t >> DIGIT_BIT); sc2 = 0;                      \
   } while (0);
\end{verbatim}
This computes a product and stores it in the ``secondary'' carry registers $\left < sc0, sc1, sc2 \right >$.

\begin{verbatim}
#define SQRADDAC(i, j)                                                         \
   do { fp_word t;                                                             \
   t = sc0 + ((fp_word)i) * ((fp_word)j);  sc0 = t;                            \
   t = sc1 + (t >> DIGIT_BIT);             sc1 = t; sc2 += t >> DIGIT_BIT;     \
   } while (0);
\end{verbatim}
This computes a product and adds it to the ``secondary'' carry registers.

\begin{verbatim}
#define SQRADDDB                                                               \
   do { fp_word t;                                                             \
   t = ((fp_word)sc0) + ((fp_word)sc0) + c0; c0 = t;                                                 \
   t = ((fp_word)sc1) + ((fp_word)sc1) + c1 + (t >> DIGIT_BIT); c1 = t;                              \
   c2 = c2 + ((fp_word)sc2) + ((fp_word)sc2) + (t >> DIGIT_BIT);                                     \
   } while (0);
\end{verbatim}
This doubles the ``secondary'' carry registers and adds the sum to the main carry registers.  Really complicated.

\section{Montgomery with Comba}
Montgomery reduction is used in modular exponentiation and is most called function during
that operation.  It's important to make sure this routine is very fast or all is lost.

Unlike the two other comba routines this one does not use a single three--digit carry 
system.  It does have three--digit carries except that the routine steps through them
in the inner loop.  This means you cannot alias them to registers (at all).

To make matters simple though the three arrays of carries are stored in one array.  The 
``c0'' array resides in $c[0 \ldots OFF1-1]$, ``c1'' in $c[OFF1 \ldots OFF2-1]$ and ``c2'' in
$c[OFF2 \ldots OFF2+FP\_SIZE-1]$.  

\begin{verbatim}
#define MONT_START 
\end{verbatim}

This allows you to insert anything at the start that you need.

\begin{verbatim}
#define MONT_FINI
\end{verbatim}

This allows you to insert anything at the end that you need.

\begin{verbatim}
#define LOOP_START \
   mu = c[x] * mp;
\end{verbatim}

This computes the $\mu$ value for the inner loop.  You can safely alias $mu$ and $mp$ to
a register if you want.

\begin{verbatim}
#define INNERMUL                                      \
   do { fp_word t;                                    \
   _c[0] = t  = ((fp_word)_c[0] + (fp_word)cy) +      \
                (((fp_word)mu) * ((fp_word)*tmpm++)); \
   cy = (t >> DIGIT_BIT);                             \
   } while (0)
\end{verbatim}

This computes the inner product and adds it to the destination and carry variable $cy$.
This uses the $mu$ value computed above (can be in a register already) and the 
$cy$ which is a chaining carry.  Inside the INNERMUL loop the $cy$ value can be kept
inside a register (hint: it always starts as $cy = 0$ in the first iteration).

Upon completion of the inner loop the macro LOOP\_END is called which is used to fetch
$cy$ into the variable the C program can see.  This is where, if you cached $cy$ in a
register you would copy it to the locally accessible C variable.

\begin{verbatim}
#define PROPCARRY \
   do { fp_digit t = _c[0] += cy; cy = (t < cy); } while (0)
\end{verbatim}

This propagates the carry upwards by one digit.  

\documentclass[b5paper]{book}
\usepackage{hyperref}
\usepackage{makeidx}
\usepackage{amssymb}
\usepackage{color}
\usepackage{alltt}
\usepackage{graphicx}
\usepackage{layout}
\def\union{\cup}
\def\intersect{\cap}
\def\getsrandom{\stackrel{\rm R}{\gets}}
\def\cross{\times}
\def\cat{\hspace{0.5em} \| \hspace{0.5em}}
\def\catn{$\|$}
\def\divides{\hspace{0.3em} | \hspace{0.3em}}
\def\nequiv{\not\equiv}
\def\approx{\raisebox{0.2ex}{\mbox{\small $\sim$}}}
\def\lcm{{\rm lcm}}
\def\gcd{{\rm gcd}}
\def\log{{\rm log}}
\def\ord{{\rm ord}}
\def\abs{{\mathit abs}}
\def\rep{{\mathit rep}}
\def\mod{{\mathit\ mod\ }}
\renewcommand{\pmod}[1]{\ ({\rm mod\ }{#1})}
\newcommand{\floor}[1]{\left\lfloor{#1}\right\rfloor}
\newcommand{\ceil}[1]{\left\lceil{#1}\right\rceil}
\def\Or{{\rm\ or\ }}
\def\And{{\rm\ and\ }}
\def\iff{\hspace{1em}\Longleftrightarrow\hspace{1em}}
\def\implies{\Rightarrow}
\def\undefined{{\rm ``undefined"}}
\def\Proof{\vspace{1ex}\noindent {\bf Proof:}\hspace{1em}}
\let\oldphi\phi
\def\phi{\varphi}
\def\Pr{{\rm Pr}}
\newcommand{\str}[1]{{\mathbf{#1}}}
\def\F{{\mathbb F}}
\def\N{{\mathbb N}}
\def\Z{{\mathbb Z}}
\def\R{{\mathbb R}}
\def\C{{\mathbb C}}
\def\Q{{\mathbb Q}}
\definecolor{DGray}{gray}{0.5}
\newcommand{\emailaddr}[1]{\mbox{$<${#1}$>$}}
\def\twiddle{\raisebox{0.3ex}{\mbox{\tiny $\sim$}}}
\def\gap{\vspace{0.5ex}}
\makeindex
\begin{document}
\frontmatter
\pagestyle{empty}
\title{TomsFastMath User Manual \\ v0.13.0}
\author{Tom St Denis \\ tomstdenis@gmail.com}
\maketitle
This text and library are all hereby placed in the public domain.  This book has been formatted for B5 
[176x250] paper using the \LaTeX{} {\em book} macro package.

\vspace{13cm}

\begin{flushleft}This project was sponsored in part by  

Secure Science Corporation \url{http://www.securescience.net}.
\end{flushleft}

\tableofcontents
\listoffigures
\mainmatter
\pagestyle{headings}
\chapter{Introduction}
\section{What is TomsFastMath?}

TomsFastMath is meant to be a very fast yet still fairly portable and easy to port large
integer arithmetic library written in ISO C.  The goal specifically is to be able to perform
very fast modular exponentiations and other related functions required for ECC, DH and RSA
cryptosystems.

Most of the library is pure ISO C portable source code while a small portion (three files) contain
a mixture of ISO C and assembler inline fragments.  Compared to LibTomMath this new library is 
meant to be much faster while sacrificing flexibiltiy.  This is accomplished through several means.

\begin{enumerate}
   \item The new code is slightly messier and contains asm blocks.
   \item This uses fixed not multiple precision integers.  
   \item It is designed only for fast modular exponentiations [e.g. less flexibility].
\end{enumerate}

To mitigate some of the problems that arise from using assembler it has been carefully and 
appropriately used where it would make the most gain in performance.  Also we use macro's
for assembler code which allows new ports to be inserted easily.

The new code uses fixed precision arithmetic which means at compile time you choose a maximum 
precision and all numbers are limited to that.  This has the benefit of not requiring any
memory heap operations (which are slow) in any of the functions.  It has the downside that 
integers that are too large are truncated.

The goal of this library is to be able to perform modular exponentiations (with an odd modulus) very
fast.  This is what takes the most time in systems such as RSA and DH.  This also requires
fast multiplication and squaring and has the side effect of speeding up ECC operations as well.

\section{License}
TomsFastMath is public domain.

\section{Building}
To build the library simply type ``make''.  Or to install in typical *unix like directories use
``make install''.  Similarly a shared library can be built with ``make -f makefile.shared install''.

You can build the test program with ``make test''.  To perform simple static testing (useful to 
test out new assembly ports) use the stest program.  Type ``make stest'' and run it on your 
target.  The program will perform three multiplications, squarings and montgomery reductions.  
Likely if your assembly code is invalid this code will exhibit the bug.

\subsection{Intel CC}
In theory you should be able to build the library with

\begin{verbatim}
CFLAGS="-O3 -ip" CC=icc make IGNORE_SPEED=1
\end{verbatim}

However, Intels inline assembler is way less advanced than GCCs.  As a result it doesn't compile.  
Fortunately it doesn't really matter.

\subsection{MSVC}
The library doesn't build with MSVC.  Imagine that.

\subsection{Build Limitations}
TomsFastMath has the following build requirements which are non--portable but under most 
circumstances not problematic.

\begin{enumerate}
\item ``CHAR\_BIT'' must be eight.  
\item The ``fp\_digit'' type must be a multiple of eight bits long.
\item The ``fp\_word'' must be at least twice the length of fp\_digit.
\end{enumerate}

\subsection{Optimization Configuration}
By default TFM is configured for 32--bit digits using ISO C source code.  This mode while portable
is not very efficient.  While building the library (from scratch) you can define one of 
several ``CFLAGS'' defines.

For example, to build with with SSE2 optimizations type 

\begin{verbatim}
CFLAGS=-DTFM_SSE2 make clean libtfm.a
\end{verbatim}

\subsubsection{x86--32}  The ``x86--32'' mode is defined by ``TFM\_X86'' and covers all
i386 and beyond processors.  It requires GCC to build and only works with 32--bit digits.  In this 
mode fp\_digit is 32--bits and fp\_word is 64--bits.  This mode will be autodetected when building 
with GCC to an  ``i386'' target.  You can override this behaviour by defining TFM\_NO\_ASM or 
another optimization mode (such as SSE2).

\subsubsection{SSE2} The ``SSE2'' mode is defined by ``TFM\_SSE2'' and requires a Pentium 4, Pentium
M or Athlon64 processor.  It requires GCC to build.  Note that you shouldn't define both
TFM\_X86 and TFM\_SSE2 at the same time.   This mode only works with 32--bit digits.  In this 
mode fp\_digit is 32--bits and fp\_word is 64--bits.  While this mode will work on the AMD Athlon64 
series of processors it is less efficient than the native ``x86--64'' mode and not recommended.

There is an additional ``TFM\_PRESCOTT'' flag that you can define for P4 Prescott processors.  This causes
the mul/sqr functions to use x86\_32 and the montgomery reduction to use SSE2 which is (so far) the fastest
combination.  If you are using an older (e.g. Northwood) generation P4 don't define this.

\subsubsection{x86--64}  The ``x86--64'' mode is defined by ``TFM\_X86\_64'' and requires a 
``x86--64'' capable processor (Athlon64 and future Pentium processors).  It requires GCC to
build and only works with 64--bit digits.  Note that by enabling this mode it will automatically
enable 64--bit digits.  In this mode fp\_digit is 64--bits and fp\_word is 128--bits.  This mode will
be autodetected when building with GCC to an ``x86--64'' target.  You can override this behaviour by defining
TFM\_NO\_ASM.

\subsubsection{ARM}  The ``ARM'' mode is defined by ``TFM\_ARM'' and requires a ARMv4 with the M instructions (enhanced 
multipliers) or higher processor.  It requires GCC and works with 32--bit digits.  In this mode fp\_digit is 32--bits and 
fp\_word is 64--bits.

\subsubsection{PPC32} The ``PPC32'' mode is defined by ``TFM\_PPC32'' and requires a standard PPC processor.  It doesn't 
use altivec or other extensions so it should work on all compliant implementations of PPC.  It requires GCC and works
with 32--bit digits.  In this mode fp\_digit is 32--bits and fp\_word is 64--bits.

\subsubsection{PPC64} The ``PPC64'' mode is defined by ``TFM\_PPC64'' and requires a 64--bit PPC processor.  

\subsubsection{AVR32} The ``AVR32'' mode is defined by ``TFM\_AVR32'' and requires an Atmel AVR32 processor.  

\subsubsection{Future Releases}  Future releases will support additional platform optimizations.
Developers of MIPS and SPARC platforms are encouraged to submit GCC asm inline patches 
(see chapter \ref{chap:asmops} for more information).

\begin{figure}[here]
\begin{small}
\begin{center}
\begin{tabular}{|l|l|}
\hline \textbf{Processor} & \textbf{Recommended Mode} \\
\hline All 32--bit x86 platforms  & TFM\_X86 \\
\hline Pentium 4                  & TFM\_SSE2 \\
\hline Pentium 4 Prescott         & TFM\_SSE2 + TFM\_PRESCOTT \\
\hline Athlon64                   & TFM\_X86\_64 \\
\hline ARMv4 or higher with M     & TFM\_ARM \\
\hline G3/G4 (32-bit PPC)         & TFM\_PPC32 \\
\hline G5 (64-bit PPC)            & TFM\_PPC64 \\
\hline Atmel AVR32                & TFM\_AVR32 \\
\hline &\\
\hline x86--32 or x86--64 (with GCC) & Leave blank and let autodetect work \\
\hline
\end{tabular}
\caption{Recommended Build Modes}
\end{center}
\end{small}
\end{figure}

\subsection{Build Configurations}
TomsFastMath is configurable in terms of which unrolled code (if any) is included.  By default, the majority of the code is included which
results in large binaries.  The first flag to try out is TFM\_ALREADY\_SET which tells TFM to turn off \textbf{all} unrolled code.  This will
result in a smaller library but also a much slower library.

From this clean state, you can start enabling unrolled code for given cryptographic tasks at hand.  A series of TFM\_MULXYZ and TFM\_SQRXYZ macros
exist to enable specific unrolled code.  For instance, TFM\_MUL32 will enable a 32 digit unrolled multiplier.  For a complete list see the tfm.h header
file.  Keep in mind this is for digits not bits.  For example, you should enable TFM\_MUL16 if you are doing 1024-bit exptmods on a 64--bit platform, enable
TFM\_MUL32 on 32--bit platforms.

To help developers use ECC there are a set of defines for the five NIST curve sizes.  They are named TFM\_ECCXYZ where XYZ is one of 192, 224, 256, 384, or 521.  These
enable the multipliers and squaring code for a given curve, autodetecting 64--bit platforms as well.  

\subsection{Precision Configuration}
The precision of all integers in this library are fixed to a limited precision.  Essentially
the rule of setting the precision is if you plan on doing modular exponentiation with $k$--bit
numbers than the precision must be fixed to $2k$--bits plus four digits.  

This is changed by altering the value of ``FP\_MAX\_SIZE'' in tfm.h to your desired size.  By default, 
the library is configured to handle upto 2048--bit inputs to the modular exponentiator.  

\chapter{Getting Started}
\section{Data Types}
TomsFastMath is a large fixed precision integer library.  It provides the functionality to 
manipulate large signed integers through a relatively trivial api and a single data type.

The ``fp\_int'' or fixed precision integer is the data type that the functions operate with.  

\begin{verbatim}
typedef struct {
    fp_digit dp[FP_SIZE];
    int      used, 
             sign;
} fp_int;
\end{verbatim}

The \textbf{dp} member is the array of digits that forms the number.  It must always be zero 
padded.  The \textbf{used} member is the count of digits used in the array.  Although the 
precision is fixed the algorithms are still tuned to not process the entire array if it 
does not have to.  The \textbf{sign} indicates the sign of the integer.  It is \textbf{FP\_ZPOS} (0)
if the integer is zero or positive and \textbf{FP\_NEG} (1) otherwise.

\section{Initialization}
\subsection{Simple Initialization}
To initialize an integer to the default state of zero use the fp\_init() function.

\index{fp\_init}
\begin{verbatim}
void fp_init(fp_int *a);
\end{verbatim}

This will initialize the fp\_int $a$ to zero.  Note that the function fp\_zero() is an alias
for fp\_init().

\subsection{Initialize Small Constants}
To initialize an integer with a small single digit value use the fp\_set() function.

\index{fp\_set}
\begin{verbatim}
void fp_set(fp_int *a, fp_digit b);
\end{verbatim}

This will initialize $a$ and set it equal to the digit $b$.  

\subsection{Initialize Copy}
To initialize an integer with a copy of another integer use the fp\_init\_copy() function.

\index{fp\_init\_copy}
\begin{verbatim}
void fp_init_copy(fp_int *a, fp_int *b)
\end{verbatim}

This will initialize $a$ as a copy of $b$.  Note that for compatibility with LibTomMath the function
fp\_copy() is also provided.

\chapter{Arithmetic Operations}
\section{Odds and Evens}
To quickly and easily tell if an integer is zero, odd or even use the following functions.

\index{fp\_iszero} \index{fp\_iseven} \index{fp\_isodd}
\begin{verbatim}
int fp_iszero(fp_int *a);
int fp_iseven(fp_int *a);
int fp_isodd(fp_int *a);
\end{verbatim}

These will return \textbf{FP\_YES} if the answer to their respective questions is yes.  Otherwise they
return \textbf{FP\_NO}.  Note that these are implemented as macros and as such you should avoid using 
++ or --~-- operators on the input operand.

\section{Sign Manipulation}
To negate or compute the absolute of an integer use the following functions.

\index{fp\_neg} \index{fp\_abs}
\begin{verbatim}
void fp_neg(fp_int *a, fp_int *b);
void fp_abs(fp_int *a, fp_int *b);
\end{verbatim}
This will compute the negation (or absolute) of $a$ and store the result in $b$.  Note that these 
are implemented as macros and as such you should avoid using ++ or --~-- operators on the input 
operand.

\section{Comparisons}
To perform signed or unsigned comparisons use following functions.

\index{fp\_cmp} \index{fp\_cmp\_mag}
\begin{verbatim}
int fp_cmp(fp_int *a, fp_int *b);
int fp_cmp_mag(fp_int *a, fp_int *b);
\end{verbatim}
These will compare $a$ to $b$.  They will return \textbf{FP\_GT} if $a$ is larger than $b$, 
\textbf{FP\_EQ} if they are equal and \textbf{FP\_LT} if $a$ is less than $b$.

The function fp\_cmp performs signed comparisons while the other performs unsigned comparisons.

\section{Shifting}
To shift the digits of an fp\_int left or right use the following functions.

\index{fp\_lshd} \index{fp\_rshd}
\begin{verbatim}
void fp_lshd(fp_int *a, int x);
void fp_rshd(fp_int *a, int x);
\end{verbatim}

These will shift the digits of $a$ left (or right respectively) $x$ digits.  

To shift individual bits of an fp\_int use the following functions.

\index{fp\_div\_2d} \index{fp\_mod\_2d} \index{fp\_mul\_2d} \index{fp\_div\_2} \index{fp\_mul\_2}
\begin{verbatim}
void fp_div_2d(fp_int *a, int b, fp_int *c, fp_int *d);
void fp_mod_2d(fp_int *a, int b, fp_int *c);
void fp_mul_2d(fp_int *a, int b, fp_int *c);
void fp_mul_2(fp_int *a, fp_int *c);
void fp_div_2(fp_int *a, fp_int *c);
void fp_2expt(fp_int *a, int b);
\end{verbatim}
fp\_div\_2d() will divide $a$ by $2^b$ and store the quotient in $c$ and remainder in $d$.  Either of 
$c$ or $d$ can be \textbf{NULL} if their value is not required.  fp\_mod\_2d() is a shortcut to 
compute the remainder directly.  fp\_mul\_2d() will multiply $a$ by $2^b$ and store the result in $c$.  

The fp\_mul\_2() and fp\_div\_2() functions are optimized multiplication and divisions by two.  The 
function fp\_2expt() will compute $a = 2^b$ quickly.

To quickly count the number of least significant bits that are zero use the following function.

\index{fp\_cnt\_lsb}
\begin{verbatim}
int fp_cnt_lsb(fp_int *a);
\end{verbatim}
This will return the number of adjacent least significant bits that are zero.  This is equivalent 
to the number of times two evenly divides $a$.

\section{Basic Algebra}

The following functions round out the basic algebraic functionality of the library.

\index{fp\_add} \index{fp\_sub} \index{fp\_mul} \index{fp\_sqr} \index{fp\_div} \index{fp\_mod}
\begin{verbatim}
void fp_add(fp_int *a, fp_int *b, fp_int *c);
void fp_sub(fp_int *a, fp_int *b, fp_int *c);
void fp_mul(fp_int *a, fp_int *b, fp_int *c);
void fp_sqr(fp_int *a, fp_int *b);
int fp_div(fp_int *a, fp_int *b, fp_int *c, fp_int *d);
int fp_mod(fp_int *a, fp_int *b, fp_int *c);
\end{verbatim}

The functions fp\_add(), fp\_sub() and fp\_mul() perform their respective operations on $a$ and
$b$ and store the result in $c$.  The function fp\_sqr() computes $b = a^2$ and is faster than
using fp\_mul() to perform the same operation.

The function fp\_div() divides $a$ by $b$ and stores the quotient in $c$ and remainder in $d$.  Either 
of $c$ and $d$ can be \textbf{NULL} if the result is not required.  The function fp\_mod() is a simple 
shortcut to find the remainder.

\section{Modular Exponentiation}
To compute a modular exponentiation use the following function.

\index{fp\_exptmod}
\begin{verbatim}
int fp_exptmod(fp_int *a, fp_int *b, fp_int *c, fp_int *d);
\end{verbatim}
This computes $d \equiv a^b \mbox{ (mod }c\mbox{)}$ for any odd $c$ and $b$.  $b$ may be negative so long as 
$a^{-1} \mbox{ (mod }c\mbox{)}$ exists.  The initial value of $a$ may be larger than $c$.  The size of $c$ must be 
half of the maximum precision used during the build of the library.  For example, by default $c$ must be less 
than $2^{2048}$.  

\section{Number Theoretic}

To perform modular inverses, greatest common divisor or least common multiples use the following
functions.

\index{fp\_invmod} \index{fp\_gcd} \index{fp\_lcm}
\begin{verbatim}
int fp_invmod(fp_int *a, fp_int *b, fp_int *c);
void fp_gcd(fp_int *a, fp_int *b, fp_int *c);
void fp_lcm(fp_int *a, fp_int *b, fp_int *c);
\end{verbatim}

The fp\_invmod() function will find the modular inverse of $a$ modulo an odd modulus $b$ and store
it in $c$ (provided it exists).  The function fp\_gcd() will compute the greatest common
divisor of $a$ and $b$ and store it in $c$.  Similarly the fp\_lcm() function will compute
the least common multiple of $a$ and $b$ and store it in $c$.

\section{Prime Numbers}
To quickly test a number for primality call this function.

\index{fp\_isprime}
\index{fp\_isprime\_ex}
\begin{verbatim}
int fp_isprime_ex(fp_int *a, int t);
\end{verbatim}
This will return \textbf{FP\_YES} if $a$ is probably prime.  It uses 256 trial divisions and
$t$ rounds of Rabin-Miller testing.  Note that this routine performs modular exponentiations
which means that $a$ must be in a valid range of precision.
The valid range of $t$ is $1 \ldots 256$.

For backwards compatibility reasons the API function fp\_isprime(a) is still available
and simply calls fp\_isprime\_ex(a, 8).

\chapter{Porting TomsFastMath}
\label{chap:asmops}
\section{Getting Started}
Porting TomsFastMath to a given processor target is usually a simple procedure.  For the most part 
assembly is used to get around the lack of a ``add with carry'' operation in the C language.  To
make matters simpler the use of assembler is through macro blocks.

Each ``port'' is defined by a block of code that re-defines the portable ISO C macros with assembler
inline blocks.  To add a new port you must designate a TFM\_XXX define that will enable your 
port when built.

\section{Multiply with Comba}
The file ``fp\_mul\_comba.c'' is responsible for providing the fast multiplication within the 
library.  This comba multiplication is fairly simple.  It uses a sliding three digit carry 
system with the variables $c0$, $c1$, $c2$.  For every digit of output $c0$ is the what will
be that digit, $c1$ will carry into the next digit and $c2$ will be the ``c1'' carry for
the next digit.  For every ``next'' digit effectively $c0$ is stored as output, $c1$ moves into
$c0$, $c2$ into $c1$ and zero into $c2$.

The following macros define the assmebler interface to the code.

\begin{verbatim}
#define COMBA_START 
\end{verbatim}

This is issued at the beginning of the multiplication function.  This is in place to allow you to
initialize any registers or machine words required.  You can leave it blank if you do not need 
it.

\begin{verbatim}
#define COMBA_CLEAR \
   c0 = c1 = c2 = 0;
\end{verbatim}

This clears the three comba carries.  If you are going to place carries in registers then 
zero the appropriate registers.  Note that the functions do not use $c0$, $c1$ or $c2$ directly
so you are free to ignore these varibles and use registers directly.

\begin{verbatim}
#define COMBA_FORWARD \
   c0 = c1; c1 = c2; c2 = 0;
\end{verbatim}

This propagates the carries after a digit has been produced.  

\begin{verbatim}
#define COMBA_STORE(x) \
   x = c0;
\end{verbatim}

This stores the $c0$ digit in the memory location specified by $x$.  Note that if you manually 
aliased $c0$ with a register than just store that register in $x$.  

\begin{verbatim}
#define COMBA_STORE2(x) \
   x = c1;
\end{verbatim}

This stores the $c1$ digit in the memory location specified by $x$.  Note that if you manually 
aliased $c1$ with a register than just store that register in $x$.  

\begin{verbatim}
#define COMBA_FINI
\end{verbatim}

If at the end of the function you need to perform some action fill this macro in. 

\begin{verbatim}
#define MULADD(i, j)                                          \
   t  = ((fp_word)i) * ((fp_word)j);                          \
   c0 = (c0 + t);              if (c0 < ((fp_digit)t))  ++c1; \
   c1 = (c1 + (t>>DIGIT_BIT)); if (c1 < (t>>DIGIT_BIT)) ++c2;
\end{verbatim}

This macro performs the ``multiply and add'' step that is central to the comba
multiplier.  It multiplies the fp\_digits $i$ and $j$ to produce a fp\_word result.  Effectively
the double--digit value is added to the three-digit carry formed by $c0$, $c1$, $c2$ where $c0$
is the least significant digit.

\section{Squaring with Comba}
Squaring is similar to multiplication except that it uses a special ``multiply and add twice'' macro
that replaces multiplications that are not required.

\begin{verbatim}
#define COMBA_START
\end{verbatim}

This allows for any initialization code you might have.  

\begin{verbatim}
#define CLEAR_CARRY \
   c0 = c1 = c2 = 0;
\end{verbatim}

This will clear the carries.  Like multiplication you can safely alias the three carry variables
to registers if you can/want to.

\begin{verbatim}
#define COMBA_STORE(x) \
   x = c0;
\end{verbatim}

Store the $c0$ carry to a given memory location.

\begin{verbatim}
#define COMBA_STORE2(x) \
   x = c1;
\end{verbatim}

Store the $c1$ carry to a given memory location.

\begin{verbatim}
#define CARRY_FORWARD \
   c0 = c1; c1 = c2; c2 = 0;
\end{verbatim}

Forward propagate all three carry variables.

\begin{verbatim}
#define COMBA_FINI
\end{verbatim}

If you need to clean up at the end of the function.

\begin{verbatim}
/* multiplies point i and j, updates carry "c1" and digit c2 */
#define SQRADD(i, j)                       \
   t  = ((fp_word)i) * ((fp_word)j);       \
   c0 = (c0 + t);              if (c0 < ((fp_digit)t))  ++c1; \
   c1 = (c1 + (t>>DIGIT_BIT)); if (c1 < (t>>DIGIT_BIT)) ++c2; 
\end{verbatim}

This is essentially the MULADD macro from the multiplication code.

\begin{verbatim}
/* for squaring some of the terms are doubled... */
#define SQRADD2(i, j)                       \
   t  = ((fp_word)i) * ((fp_word)j);       \
   c0 = (c0 + t);              if (c0 < ((fp_digit)t))  ++c1; \
   c1 = (c1 + (t>>DIGIT_BIT)); if (c1 < (t>>DIGIT_BIT)) ++c2; \
   c0 = (c0 + t);              if (c0 < ((fp_digit)t))  ++c1; \
   c1 = (c1 + (t>>DIGIT_BIT)); if (c1 < (t>>DIGIT_BIT)) ++c2; 
\end{verbatim}

This is like SQRADD except it adds the produce twice.  It's similar to 
computing SQRADD(i, j*2).

To further make things interesting the squaring code also has ``doubles'' (see my LTM book chapter five...) which are
handled with these macros.

\begin{verbatim}
#define SQRADDSC(i, j)                                                         \
   do { fp_word t;                                                             \
      t =  ((fp_word)i) * ((fp_word)j);                                        \
      sc0 = (fp_digit)t; sc1 = (t >> DIGIT_BIT); sc2 = 0;                      \
   } while (0);
\end{verbatim}
This computes a product and stores it in the ``secondary'' carry registers $\left < sc0, sc1, sc2 \right >$.

\begin{verbatim}
#define SQRADDAC(i, j)                                                         \
   do { fp_word t;                                                             \
   t = sc0 + ((fp_word)i) * ((fp_word)j);  sc0 = t;                            \
   t = sc1 + (t >> DIGIT_BIT);             sc1 = t; sc2 += t >> DIGIT_BIT;     \
   } while (0);
\end{verbatim}
This computes a product and adds it to the ``secondary'' carry registers.

\begin{verbatim}
#define SQRADDDB                                                               \
   do { fp_word t;                                                             \
   t = ((fp_word)sc0) + ((fp_word)sc0) + c0; c0 = t;                                                 \
   t = ((fp_word)sc1) + ((fp_word)sc1) + c1 + (t >> DIGIT_BIT); c1 = t;                              \
   c2 = c2 + ((fp_word)sc2) + ((fp_word)sc2) + (t >> DIGIT_BIT);                                     \
   } while (0);
\end{verbatim}
This doubles the ``secondary'' carry registers and adds the sum to the main carry registers.  Really complicated.

\section{Montgomery with Comba}
Montgomery reduction is used in modular exponentiation and is most called function during
that operation.  It's important to make sure this routine is very fast or all is lost.

Unlike the two other comba routines this one does not use a single three--digit carry 
system.  It does have three--digit carries except that the routine steps through them
in the inner loop.  This means you cannot alias them to registers (at all).

To make matters simple though the three arrays of carries are stored in one array.  The 
``c0'' array resides in $c[0 \ldots OFF1-1]$, ``c1'' in $c[OFF1 \ldots OFF2-1]$ and ``c2'' in
$c[OFF2 \ldots OFF2+FP\_SIZE-1]$.  

\begin{verbatim}
#define MONT_START 
\end{verbatim}

This allows you to insert anything at the start that you need.

\begin{verbatim}
#define MONT_FINI
\end{verbatim}

This allows you to insert anything at the end that you need.

\begin{verbatim}
#define LOOP_START \
   mu = c[x] * mp;
\end{verbatim}

This computes the $\mu$ value for the inner loop.  You can safely alias $mu$ and $mp$ to
a register if you want.

\begin{verbatim}
#define INNERMUL                                      \
   do { fp_word t;                                    \
   _c[0] = t  = ((fp_word)_c[0] + (fp_word)cy) +      \
                (((fp_word)mu) * ((fp_word)*tmpm++)); \
   cy = (t >> DIGIT_BIT);                             \
   } while (0)
\end{verbatim}

This computes the inner product and adds it to the destination and carry variable $cy$.
This uses the $mu$ value computed above (can be in a register already) and the 
$cy$ which is a chaining carry.  Inside the INNERMUL loop the $cy$ value can be kept
inside a register (hint: it always starts as $cy = 0$ in the first iteration).

Upon completion of the inner loop the macro LOOP\_END is called which is used to fetch
$cy$ into the variable the C program can see.  This is where, if you cached $cy$ in a
register you would copy it to the locally accessible C variable.

\begin{verbatim}
#define PROPCARRY \
   do { fp_digit t = _c[0] += cy; cy = (t < cy); } while (0)
\end{verbatim}

This propagates the carry upwards by one digit.  

\input{tfm.ind}

\end{document}


\end{document}


\end{document}


\end{document}
